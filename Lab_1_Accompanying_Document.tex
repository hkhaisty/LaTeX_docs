\documentclass[11pt]{article}

\usepackage[utf8]{inputenc}
\usepackage{listings}
\usepackage{color}
\usepackage{float}

\usepackage{geometry}
\geometry{a4paper}
\geometry{margin = .5in}

\usepackage{graphicx}

\definecolor{dkgreen}{rgb}{0,0.6,0}
\definecolor{gray}{rgb}{0.5,0.5,0.5}
\definecolor{mauve}{rgb}{0.58,0,0.82}
\lstset{frame=tb,
  language=C++,
  aboveskip=3mm,
  belowskip=3mm,
  showstringspaces=false,
  columns=flexible,
  basicstyle={\small\ttfamily},
  numbers=none,
  numberstyle=\tiny\color{gray},
  keywordstyle=\color{blue},
  commentstyle=\color{dkgreen},
  stringstyle=\color{mauve},
  breaklines=true,
  breakatwhitespace=true,
  tabsize=3
}


\title{Lab 1 Accompanying Document}
\author{Harry Haisty}

\begin{document}

\maketitle

\section*{The Output}
\begin{figure}[H]
\centering

\includegraphics[width=15cm, height = 15cm,keepaspectratio]{Lab_1_Snip}
\caption{Printed output for lab 1}
\end{figure}

\section*{The Description}
This program was designed to take in a set of double values, add them to a dynamically-sized list, and then use a variety of methods to determine different mathematical qualities of the contents of the list. This includes the length, or size of the list, the sum of the elements of the list, the mean of the values of the list, the minimum number of the list, and the maximum number of the list. 
\newline
The program also includes a "next" function, which allows you to hard code new values into the list, or could potentially accept user input from the console. The program also has a "clear" function written in, which drops all of the values from the list and prepares it for new inputs. 

\newpage
\section*{The Code}

\subsection*{Statistician.h}
\begin{lstlisting}
#include <iostream>
#include <vector>

using namespace std; 

class statistician
{
public:
	statistician();
	void next(double r);
	void reset();
	int length();
	double sum();
	double mean();
	double minimum();
	double maximum();
	int count;       // How many numbers in the sequence
	double total;    // The sum of all the numbers in the sequence
	double tinyest;  // The smallest number in the sequence
	double largest;  // The largest number in the sequence
private: 
	vector <double> vector_list;
	
};

\end{lstlisting}


\subsection*{Lab1.cpp}
\begin{lstlisting}
#include "stdafx.h"
#include "Statistician.h"

/*Lab 1 Data Structures
Harriet Haisty
Codename: Cheese Puff
Dr. Eyles
08/27/18*/

int main()
{
	//create new statistician
	statistician new_stat;
	
	//feed in several numbers to statistician
	new_stat.next(23);
	new_stat.next(54);
	new_stat.next(32);
	new_stat.next(92);
	new_stat.next(81);
	new_stat.next(17);

	//call length method to print
	cout << "Length: " << new_stat.length() << endl;

	//call sum method to print
	cout << "Sum: " << new_stat.sum() << endl;

	//call mean method to print
	cout << "Mean: " << new_stat.mean() << endl;

	//call minimum method to print
	cout << "Minimum: " << new_stat.minimum() << endl;

	//call maximum method to print
	cout << "Maximum: " << new_stat.maximum() << endl;

	//call reset function
	new_stat.reset();

	//add a new set of values into statistician
	new_stat.next(64);
	new_stat.next(12);
	new_stat.next(26);
	new_stat.next(42);
	new_stat.next(11);
	new_stat.next(97);

	//print a new line and call methods again to display new #'s
	cout << "\nLength: " << new_stat.length() << endl;
	cout << "Sum: " << new_stat.sum() << endl;
	cout << "Mean: " << new_stat.mean() << endl;
	cout << "Minimum: " << new_stat.minimum() << endl;
	cout << "Maximum: " << new_stat.maximum() << endl;

	//hold console window open
	cin.get();
}
\end{lstlisting}

\subsection*{Statistician.cpp}
\begin{lstlisting}
#include "stdafx.h"
#include "Statistician.h"

//default constructor
statistician::statistician()
{
 
}

void statistician::next(double r)
{
	//add double value passed in as "r" to a resizeable list, or vector list
	return vector_list.push_back(r);
}

void statistician::reset() 
{
	//drop all values from vector list
	return vector_list.clear();
}
int statistician::length() 
{
	//return number of elements inside list
	return vector_list.size(); 
}
double statistician::sum()
{
	//declare/initialize sum value
	double sum_of_elements = 0;

	//loop through and add all element values to sum value
	for (size_t i = 0; i < vector_list.size(); i++)
	{
		sum_of_elements += vector_list.at(i);
	}

	//return new number
	return sum_of_elements;
}
double statistician::mean() 
{
	//call sum method, divide sum of values by number of values
	double value = sum() / vector_list.size(); 
	return value; 
}
double statistician::minimum() 
{
	//catch in case user has not put any values into the list
	if (vector_list.size() == 0)
	{
		throw "Empty list";
	}

	//declare/initialize minimum value variable to first value in list
	double minimum_value = vector_list.at(0);

	//loop through and compare all values to minimum, if they are smaller, replace min value
	for (size_t i = 1; i < vector_list.size(); i++)
	{
		if (vector_list.at(i) < minimum_value)
		{
			minimum_value = vector_list.at(i);
		}
	}
	//return the smallest value
	return minimum_value;
}
double statistician::maximum() 
{
	//catch if there are no values in the list
	if (vector_list.size() == 0)
	{
		throw "Empty list";
	}

	//initialize maximum value to first element in the list
	double maximum_value = vector_list.at(0);

	//loop through, if a value in the list is bigger than current max value, replace max value with that value
	for (size_t i = 1; i > vector_list.size(); i++)
	{
		if (vector_list.at(i) > maximum_value)
		{
			maximum_value = vector_list.at(i);
		}
	}

	//return the biggest number
	return maximum_value;
}

//statistician::~statistician()
//{
//}

\end{lstlisting}





\end{document}