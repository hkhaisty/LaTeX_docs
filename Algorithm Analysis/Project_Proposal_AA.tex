\documentclass[12pt]{article}
\usepackage[utf8]{inputenc}
\usepackage[a4paper, total={6in, 8in}]{geometry}

\title{Pathfinding Algorithms and their Applications}
\author{Harry Haisty}
\date{March 2019}

\begin{document}
\begin{titlepage}
   \begin{center}
       \vspace*{4cm}
 
       \textbf{Pathfinding Algorithms and their Applications}
 
       \vspace{0.5cm}
        CS 4308 -- Algorithm Analysis, Professor Sharon Perry
\begin{center}
            \textit{March 2019}

\end{center}
       \vspace{.1cm} 
       Harry Haisty
       \vspace{.3cm}
       \vfill
 
       \vspace{0.8cm}
   \end{center}
\end{titlepage}


\section*{Pathfinding Algorithms \& Applications Proposal}
The focus of this project is to be on pathfinding algorithms, namely A* and Dijkstra's algorithms. The first part of the project will mainly focus on the history of the algorithms, the mathematicians or scientists whom their creation is credited to, and what problems these people were attempting to solve when they developed these algorithms. This section will focus on the article \textit{On the History of the Shortest Path Problem} by Schrijver
\newline \newline
The second part of this project will focus on the description of these algorithms, will talk about how they work to solve problems, and will display pseudocode to help readers understand a little bit more about how the algorithms are set up. It will go into detail about the complexity of the algorithms, and their merits and drawbacks in certain use cases. This section will reference the IEEE published article entitled \textit{Best Routes Selection Using Dijkstra and Floyd-Warshall Algorithm} by Risald, Mirino, and Suyoto, and 
\newline \newline
The third part of this project will focus on the past problems that have been solved by using these algorithms, and whether modifications to these algorithms have occurred. It will touch on ideas such as past implementations of these algorithms, and how it changed that technology. This portion of the paper will reference the ACM published article entitled \textit{Analog Circuit Shielding Routing Algorithm Based on Net Classification} by Gao, Shen, Cai, and Yao
\newline \newline
The final part of the project will focus on current applications of these algorithms. It will determine why certain technology that uses these algorithms is superior to homologous technologies that use the same algorithm. It will reference content discussed in the ACM published article \textit{Dijkstra's Algorithm and Google Maps} by Lanning, Harrell, and Wang, and the IEEE published article entitled \textit{Smart Traffic Systems: Dynamic A* Traffic in GIS Driving Paths Applications} by Halaoui.


\end{document}
