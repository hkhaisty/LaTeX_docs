\documentclass[11pt]{article}

    \usepackage[utf8]{inputenc}
    \usepackage{listings}
    \usepackage{kotex}
    \usepackage{color}
    \usepackage{float}
    \usepackage{hyperref}
    \usepackage{geometry}
    \geometry{a4paper}
    \geometry{margin = .5in}
    
    \usepackage{graphicx}
    
    \title{Intro to Software Engineering: Assignment 10}
    \author{Harry Haisty}
    
    \begin{document}
    \maketitle
    \begin{enumerate}
        \item Consider the diagram shown in \textit{Figure 10.8}
        \begin{enumerate}
        \item How many logical paths are there? List them all:
        
        \item How many paths are required to cover all the statements?
        
        \item How many paths are required to cover all the branches?
        \end{enumerate}
        
        \item In code inspection, what would you set as the condition (e.g, how many discovered defects) for reinspection?
        \newline
        \color{blue}{If the moderator deems that in the follow-up there are too many defects, they may call for a reinspection.}
        
        \color{black}
        
        \item List the four techniques discussed to perform verification and validation.
        
        
        \item List two techniques you can use to perform validation -- that is, to ensure your program meets user requirements. 
        
        \item Briefly explain the concept of static analysis, and to which software products it can be applied.
        
        \item Consider the simple case of testing 2 variables, \textit{X} and \textit{Y}, where \textit{X} must be a non-negative number, and \textit{Y} must be aa number between -5 and +15. Utilizing boundary value analysis, list the test cases. 
        
        \item Describe the steps involved in a formal inspection process and the role of a moderator in this process.
        \begin{itemize}
        \item \textit{Planning: } a team and a moderator are selected and given the proper materials before the actual inspection meeting.
        \item \textit{Overview: } a high-level overview of the work product is presented.
        \item \textit{Preparation: } every member of the team is expected to study up before the actual work begins.
        \item \textit{Examination: } an actual meeting is arranged where all the inspectors gather to review the product. The author is not allowed to read the work off to the team, so another reader is designated. People search for defects. 
        \item \textit{Rework: } the author corrects all the defects found in the work. 
        \item \textit{Follow-Up: } The corrections are checked by the moderator, and the corrected work is inspected again.
        \end{itemize}
        
        \item What is the difference between performance testing and stress testing? 
        
        
    \end{enumerate}
    
    \end{document}