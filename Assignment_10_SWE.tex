\documentclass[11pt]{article}

    \usepackage[utf8]{inputenc}
    \usepackage{listings}
    \usepackage{kotex}
    \usepackage{color}
    \usepackage{float}
    \usepackage{hyperref}
    \usepackage{graphicx}
    \usepackage{geometry}
    \geometry{a4paper}
    \geometry{margin = .5in}    

    \title{Intro to Software Engineering: Assignment 10}
    \author{Harry Haisty}
    
    \begin{document}
    \maketitle
    \begin{enumerate}
        \item Consider the diagram shown in \textit{Figure 10.8}
        \begin{enumerate}
        \item How many logical paths are there? List them all:
        
        \item How many paths are required to cover all the statements?
        
        \item How many paths are required to cover all the branches?
        \end{enumerate}
        
        \item In code inspection, what would you set as the condition (e.g, how many discovered defects) for reinspection?
        
        \item List the four techniques discussed to perform verification and validation.
        
        \item List two techniques you can use to perform validation -- that is, to ensure your program meets user requirements. 
        
        \item Briefly explain the concept of static analysis, and to which software products it can be applied.
        
        \item Consider the simple case of testing 2 variables, \textit{X} and \textit{Y}, where \textit{X} must be a non-negative number, and \textit{Y} must be aa number between -5 and +15. Utilizing boundary value analysis, list the test cases. 
        
        \item Describe the steps involved in a formal inspection process and the role of a moderator in this process.
        \begin{itemize}
        \item \textit{Planning: } add some text here
        \item \textit{Overview: }
        \item \textit{Preparation: } 
        \item \textit{Examination: } 
        \item \textit{Rework: }
        \item \textit{Follow-Up: } 
        \end{itemize}
        
        \item What is the difference between performance testing and stress testing? 
        
        
    \end{enumerate}
    
    \end{document}