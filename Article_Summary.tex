\documentclass[11pt]{article}
\usepackage[utf8]{inputenc}
\usepackage{listings}
\usepackage{color}
\usepackage{float}
\usepackage{geometry}
\geometry{a4paper}
\geometry{margin = .5in}

\title{Article Review}
\author{Harry Haisty}

\begin{document}

\maketitle

This article dicusses the use of the same device as we are using in our current research.The device includes a gold plated electrode that is covered in a conductive liquid to read brain waves. The object of this research project, however, was to determine the expertise of programmers. Most often, expertise cannot be measured without extensive testing, and potential employers rely heavily on the potential employee's own word to verify their level of comfort with a language or technology. Each participant was given a series of three simple java programs, all which involved for loops. The participant is prompted to relax by a blank screen that displays the word "Relax" for 10 seconds before displaying the Java challenge.
\newline \newline
Each challenge the participant was presented with was preceded by a 10-second delay, during which the participant was prompted to relax. The challenged grew progressively harder. There was a ternary value of responses to the prompts: Correct, Incorrect, or Skipped. This study divided the participants by ``Class Level," Class 0 being the participants who had taken no or the bare minimum amount of Computer Science courses to participate in the study, and Class 4 being comprised of students who had already taken 4000 level classes. Out of  120 tries, the level 0 participants only got 14 correct, cumulatively. Out of 105, the Class 4 students got 76 correct. 
\newline\newline
The conclusion of this paper was that students with more coarseload under their belts exhibited lower working memory load, due to their years of experience with the material. The students who were earlier in their degrees exhibited higher memory load, as they were wracking their brains for relevant information so that they could be able to accomplish the challenges. 

\end{document}