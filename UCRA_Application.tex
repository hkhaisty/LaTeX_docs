\documentclass{article}
\usepackage[utf8]{inputenc}
\usepackage{geometry}
\usepackage{float}
\usepackage{graphicx}
\geometry{margin = .5in}

\title{UCRA Application}
\author{Harry Haisty}
\date{October 2018}

\begin{document}

\section*{Faculty Member Information}

\textbf{Faculty Member Name:} Dr. Sarah North \newline
\textbf{Faculty Member E-mail:} snorth@kennesaw.edu \newline
\textbf{Faculty Member Department:} Computer Science 

\section*{Faculty Member Interest} 
My primary philosophy of research and mentoring students for over 2 decades has been subscribing to student-centered education: believing that for the students to learn, the students must be comfortable in the learning environment, allowing students to freely engage in research project activities in a variety of contemporary technologies.  I believe effective research and mentoring undergraduate students must respect the learner’s self-concept and self-efficacy.  To be effective, mentors must care deeply about their learners and be committed to the whole person, while striving for a balance between challenging people to do their best and nurturing their efforts to be successful.
\newline \newline
I have mentored over 60 undergraduate students in building skills and conducting mainstream research. This has included STEM NSF/LSAMP, Capstone students, and High School Internships, and I have also been directly involved with several undergraduate students in their research projects at KSU. Since my residency at KSU, I have written and submitted three proposals mentoring undergraduate students on research in collaborative efforts with CS colleagues submitted to the NSF, CARET, and Mentor-Protégé.  
\newline\newline
We were able to publish two of those projects: DRIP–Data Rich, Information Poor: A Concise Synopsis of Data Mining, Universal Journal of Management, Horizon Research Publishing, USA, 2015, and CS++: Expending CS Curriculum via Open Cybersecurity Courseware at 2014 51st ACM-Southeast Conference. The CARET and Mentor-Protégé proposals, entitled “A Study on Malicious Android Applications,” were not accepted due to the limited CSM acceptance rating of 30\% funding; however, in both the CARET and Mentor-Protégé projects, a few students collaborated with us and had their research published in peer-reviewed journals and conference proceedings, including the 15th IEEE International Symposium on High Assurance Systems Engineering (HASE), International Journal of Secure Software Engineering (IJSSE), and International Journal of Network Security & Its Applications (IJNSA). 
\newline \newline
The most recent student publication was in April, 2018, “Performances Analysis of Brain-Computer Interface in Aerial Drone”, Annual ACM Southeast Conference, 2018 ACMSE Conference, and received the first place award at the CCSE Capstone project presentation.
\newline \newline
We  are  currently  building  on  the  research  project  from  the  team  I  mentored  last  year.  The research team has surveyed and collected data from over 50 participants, and has begun to draft models to identify differences in the ability to control a 3D Graphic using recorded brain function.  I am excited to see my students grow and learn more about the research process and become invested in a group project.

\section*{Student Impact}
The current UCRA applicant, Harriet (Harry) Haisty, is the project manager on an ongoing brain-computer interface research project that utilizes the Emotiv EPOC+\texttrademark and accompanying software. This project began in August, 2018, and has gained traction, already receiving over fifty research participants. Harriet has been involved heavily in the data collection and interpretation, spending long hours at the school in order to get as many participants as possible. 
\newline \newline
This is one of the first projects completed by a college research team that uses the Emotiv EPOC+\texttrademark headsets, and has given tremendous insight so far into the possibilities offered by such a revolutionary technology. The headset used in this project is designed to read signals from the brain of the user, and has incredible potential to allow improved quality of life for people with limited mobility. Collecting data in large quantities allows the research team to analyze various dichotomies within the user population. 
\newline \newline
The analysis of these dichotomies could help companies developing this, and similar technologies, to refine their interpretation of different brain signals so that consumers with limited mobility can have a more customizable brain-computer interface, and that they can become more successful in employing the use of this technology. 
\newline \newline
The current objective of this project is to collect enough data to draw a positive connection between certain users and their success rate of at-will controlling certain functions within the software. The plan thus far is to present this poster at Kennesaw State University's Capstone Project event, but there are other conferences that would be perfect venues to present this research. The IEEE SoutheastCon 2019 is an event that members of IEEE-CS at Kennesaw State University plan to go to, and Harriet's team would like to present their paper at this conference as well as others. 



\section*{Student Information}
\textbf{Student Name:} Harriet Haisty \newline
\textbf{Student E-mail:} hhaisty@students.kennesaw.edu \newline
\textbf{Student Department:} Computer Science \newline
\textbf{Graduation Date:} December 2019

\section*{Project Description}
"Non Invasive BCI [Brain Control Interface] Devices are considered the safest type and low[est]-cost type of devices." (Ramadan, Refat, Elshahed & Ali, 2015). This non-invasive headset uses a conductive saline liquid to soak felt pads, which are connected to a gold-plated sensor on the headset. This send a continuous signal back to the software, and detects fluctuations in brain activity. It is able to detect changes in mood and facial expressions. Most importantly, this device can assign user-specified brain activity to certain tasks. This means that the interface is customizable, and that users can continue to retrain the device and record enough activity to get a successful mental command assigned to a specific task. 
\newline \newline
We begin by having the research participants sign a consent form, allowing us to use and publish their data. We then move on to a brief survey, which asks for their gender identity, their Major, their College, and their age. 
\newline \newline
The survey then moves on to a brain hemisphere-detection survey, which estimates the proportions of someone's mental hemispheric dominance. This information is then recorded in the survey.
\newline \newline
The final part of the survey is a Myers-Briggs Personality Type indicator, which asks the students to read a paragraph on dichotomous personality types and select the one that they identify with more. This gives us a better understanding of their personality, and allows us to attempt to draw a relation between immediate success of mental command association and traits such as extroversion or introversion. 
\newline \newline
This survey is done completely in Google Forms, the data is private and is only accessible by the collaborators, who have to be invited to collaborate on the project. All of the surveys are anonymous, and do not ask for any identifying information. Google Forms is also able to generate visual graphs and Comma Separated Variable files immediately based on the results of the participants. 

\begin{figure}[h]
\centering
\includegraphics[width=10cm, height = 10cm,keepaspectratio]{gender_identity}
\caption{Gender Identity Breakdown of Participants}
\end{figure}

After the survey is recorded, the felt pads on the headset are saturated with the conductive liquid and placed on the participants' heads. The sensors are checked for conductivity, and when the sensors receive positive, consistent signal, the training process begins. 
\newline \newline
The application is loaded by the investigating party with four commands, \textit{push, lift, left,} and \textit{rotate right}. These commands were selected because of their difference and increasing difficulty. After the training process is complete, the participants get 70 seconds to recall and execute the brain function that was assigned during the training session. 
\newline \newline
The users' ability to recall and execute the commands are scored by the investigator on a scale from \textit{0 -- 3}, \textit{0} being that the participant is unable to execute the assigned function at all, and 3 is that the user is able to call the function multiple times and with extreme success.  
\newline\newline 
Our project team is comprised of five college seniors and one college junior. We are all seeking Bachelor's degrees in Computer Science, with two of us looking into User Interface Engineering specifically.

\subsection*{Project References}
Ramadan, R.A., Refat, S., Elshahed, M.A., and Ali, R.A. (2015). Chapter 2: Basics of Brain Computer Interface. \textit{Brain-Computer Interfaces}, 31--48

\subsection*{Project Funding Description}
I began researching under the supervision of Dr. Sarah North in August of 2018, and through her guidance and support I have been able to reach over 50 participants in a couple short months. 
\newline \newline
I am applying for UCRA funding so that I may complete some incredible necessary repairs to the headset that we have, order replacement felt pads, and seek professional maintenance on the laptop we are using. It runs very slowly, extending the amount of time for the experiments and in a couple of cases causing the experiments to fail completely. 
\newline \newline
In addition to repairing the equipment we are using for this research, we would like to offer a small incentive for people to participate in our study. Currently, we have found it difficult to reach people on the Kennesaw campus, and the people we have been able to reach primarily are students within our college of Computing and Software Engineering. When we have been able to make it to the Kennesaw campus, we only were able to get 3 participants each time, because nobody was interested or had the time to participate in our study.

\begin{figure}[H]
    \centering
    \includegraphics[width=10cm, height = 10cm,keepaspectratio]{college_breakdown}
    \caption{Breakdown of Colleges of Participants}
    \label{fig:my_label}
\end{figure}

I am also applying for UCRA funding so that my team and I may submit and present our research and paper to different conferences and competitions so that we may represent Kennesaw State University. This research is useful to developers for non-invasive Brain-Computer Interface devices, and could go a long way in providing understanding of consistent user success.
\newpage
\subsection*{Justification}
\begin{enumerate}
    \item Emotiv EPOC+ Felt Sensor Replacements: \$79.95
    \item Pizza for encouraging research participation (Kennesaw) 
    \begin{enumerate}
        \item[] Large (14") Cheese Pizzas (3): \$32.37
        \item[] Large (14") Peperroni Pizzas (3): \$37.11
        \item[] Delivery Charge: \$3.49
        \item[] Sales Tax: \$4.38
        \item[] Delivery Tip: \$17.65
        \item[] \textbf{Total:} \textbf{\$95}
    \end{enumerate}
       
      \item Accomodations for trip to IEEE Conference
      \begin{enumerate}
          \item[] Nightly Room Cost: \$120 (Breakfast Included)
          \newline
          x 3
          \newline
        -------------
        \newline
        \$360
      \end{enumerate}
      \item Food Budget (Thursday--Sunday) IEEE Conference: \$80
\end{enumerate}
\textbf{Total Requested:} \$614.95


\end{document}
