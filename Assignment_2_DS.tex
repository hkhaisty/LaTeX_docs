\documentclass[11pt]{article}

\usepackage[utf8]{inputenc}
\usepackage{listings}
\usepackage{float}
\usepackage{xcolor}
\usepackage{geometry}

\geometry{a4paper}
\geometry{margin = .5in}

\usepackage{graphicx}

\definecolor{dkgreen}{rgb}{0,0.6,0}
\definecolor{gray}{rgb}{0.5,0.5,0.5}
\definecolor{mauve}{rgb}{0.58,0,0.82}
\lstset{frame=tb,
  language=C++,
  aboveskip=3mm,
  belowskip=3mm,
  showstringspaces=false,
  columns=flexible,
  basicstyle={\small\ttfamily},
  numberstyle=\tiny\color{gray},
  numbers=left, 
  keywordstyle=\color{orange},
  breaklines=true,
  breakatwhitespace=true,
  frame=tb, 
  tabsize=4, 
  commentstyle=\color{gray}, 
  stringstyle=\color{dkgreen} 
}

\title{Data Structures: Assignment 2}
\author{Harry Haisty}

\begin{document}
\maketitle
\centerline{Codename: Cheese Puff}

\section*{The Output}
\begin{figure}[H]
    \centering
    \includegraphics{assignment_2_1}
    \caption{Output for factorial recursion program}
    \label{fig:part 1}
\end{figure}

\begin{figure}[H]
    \centering
    \includegraphics{assignment_2_2}
    \caption{Output for sum of integers program}
    \label{fig:part 2}
\end{figure}

\begin{figure}[H]
    \centering
    \includegraphics{assignment_2_3}
    \caption{Output for Fibonacci sequence up to a value}
    \label{fig:part 3}
\end{figure}

\begin{figure}[H]
    \centering
    \includegraphics{assignment_2_4}
    \caption{Output for evaluating truth of palindrome}
    \label{fig:part 4}
\end{figure}

\begin{figure}[H]
    \centering
    \includegraphics{assignment_2_5}
    \caption{Output for printing strings backwards}
    \label{fig:part 5}
\end{figure}

\begin{figure}[H]
    \centering
    \includegraphics{assignment_2_B}
    \caption{Output for calculating difference in runtime}
    \label{fig:part 6}
\end{figure}


\section*{The Code}

\subsection*{Computing Factorials with Recursive Functions}
\begin{lstlisting}
#include <iostream>

using namespace std;

//make the type long long to accomodate the large factorial values
long long factorial(long long n) {
    //the factorial value only goes to 1, if it were multiplied by 0 then the whole value would be 0
    if (n > 1)
        //recursively decrement the value of n and multiply
        return n * factorial(n - 1);
    else
        //if the input value is 1 or less, return just the value "1"
        return 1;
}

int main() {
    //declare user input as type long long
    long long user_input;

    //prompt user to input value
    cout << "Enter an integer: ";
    //take in user input value
    cin >> user_input;

    //print out recursive call to factorial method
    cout << factorial(user_input) << endl;
}
\end{lstlisting}

\subsection*{Computing Sums with Recursive Functions}
\begin{lstlisting}
#include <iostream>

using namespace std;

//works similarly to the factorial method
int sum(int n) {
    //n must be above 0, otherwise the loop will continue negatively forever
    if (n > 0)
        return n + sum(n - 1);

        //if n is less than or equal to 0, return the value 0.
        // if you return 1, then 1 will be added to the final sum value
    else
        return 0;
}

int main() {
    //declare user input
    int user_input = 0;

    //prompt user input
    cout << "Enter an integer: ";
    //accept user input and initialize user input value
    cin >> user_input;

    //call recursive function with user input value
    cout << sum(user_input);
}
\end{lstlisting}


\subsection*{Calculating Fibonacci Sequence with Recursive Function}
\begin{lstlisting}
#include <iostream>

using namespace std;

int fibbonacci_sequence(int n) {

    //declares and initializes num1 and num2 values to 0 and 1 so that program doesn't hang
    int num1 = 0, num2 = 1;

    //stops loop if n is smaller than threshold
    if (n < 1)
        return n;

    //loops through to user input value
    for (int i = 1; i < n; i++) {
        //prints value
        printf("%d", num2, " ");
        //sets next to sum of first two integers, per the fibonacci sequence
        int next = num1 + num2;
        //sets smaller num1 to larger num2
        num1 = num2;
        //sets num2 to larger number
        num2 = next;
    }
}

int main() {

    //declares integer N
    int N;

    //prompts user to set integer N to value
    cout << "Enter an integer value: ";
    //sets integer N to user value
    cin >> N;

    //prints out call to fibonacci sequence
    cout << fibbonacci_sequence(N);
}
\end{lstlisting}

\subsection*{Evaluating Palindromes using Recursion}
\begin{lstlisting}
#include <iostream>

using namespace std;

//return boolean value to user to show if palindrome value is true or not
bool is_palindrome(const string &str, int first_value, int last_value) {
    //if first and last values left are equal, return true
    if (first_value >= last_value)
        return true;
    //if the first and last values left are not equal, return false
    if (str[first_value] != str[last_value])
        return false;
    //increment first value and decrement last value to evaluate against eachother
    return is_palindrome(str, ++first_value, --last_value);
}

int main() {
    //declare user input value
    string user_string;

    //prompt user to enter a string
    cout << "enter a string: ";
    //initialize user input value
    cin >> user_string;

    //print output based on returned value from recursive function
    if (is_palindrome(user_string, 0, user_string.length() - 1) == 1)
        cout << user_string << " is a palindrome.";
    else
        cout << user_string << " is not a palindrome";
}
\end{lstlisting}

\subsection*{Using Recursion to Print Strings Backwards}
\begin{lstlisting}
#include <iostream>

using namespace std;

string reverse_string(string input, string reverse, int n) {
    //base case, return empty string
    if (n == 0)
        return "";

    //reverse elements, append individual elements to new string
    return reverse += input.back() + reverse_string(input.substr(0, n - 1), reverse, n - 1);
}

int main() {
    //declare user input
    string user_input;

    //prompt user for input
    cout << "Enter a string: ";
    //accept user input
    cin >> user_input;

    //print out the reverse of the string by calling recursive function
    cout << reverse_string(user_input, "", user_input.length()) << endl;
}


\end{lstlisting}


\subsection*{Calculating Iterative Time vs. Recursive Time}
\begin{lstlisting}
#include <iostream>
#include <chrono>

using namespace std;

long sum;

long iterative_sum(int n) {
    for (int i = 0; i < n; ++i)
        sum += i;
}

long recursive_sum(int n) {
    if (n > 0)
        n + recursive_sum(n - 1);
}

int main() {
    int user_input = 0;
    typedef std::ratio<1l, 1000000000000l> pico;

    cout << "Enter an integer: ";
    cin >> user_input;

    auto start = chrono::steady_clock::now();
    iterative_sum(user_input);
    auto end = chrono::steady_clock::now();

    chrono::duration<double, std::pico> elapsed_time = (end - start);
    cout << elapsed_time.count() << " picoseconds" << endl;

    start = chrono::steady_clock::now();
    recursive_sum(user_input);
    end = chrono::steady_clock::now();

    chrono::duration<double, std::pico> elapsed_time_2 = (end - start);
    cout << elapsed_time_2.count() << " picoseconds";
}

\end{lstlisting}

\section*{2.3 Trees}
\begin{enumerate}
    \item List the leaves of the tree
    \begin{itemize}
        \item K, L, F, G, M, I, J
    \end{itemize}
    
    \item List the internal nodes of the tree
    \begin{itemize}
        \item A, B, C, D, E, H
    \end{itemize}
    
    \item List a subtree
    \begin{itemize}
        \item B, E, K, L
    \end{itemize}
    
    \item Name a child and its parent node
    \begin{itemize}
        \item C: \textit{parent node}, F: \textit{child node}
    \end{itemize}
    
    \item What is a branch?
    \begin{itemize}
        \item  a node that has child nodes
    \end{itemize}
    
    \item Name sibling nodes
    \begin{itemize}
        \item C and D are sibling nodes because they come from the same parent node
    \end{itemize}
    
    \item List all descendants of B
    \begin{itemize}
        \item E, K, L
    \end{itemize}
    
    \item What is the path from A to J?
    \begin{itemize}
        \item A, D, J
    \end{itemize}
    
    \item What is the root node?
    \begin{itemize}
        \item A is the root node
    \end{itemize}
    
    \item What is the depth of H?
    \begin{itemize}
        \item The depth of H is 2
    \end{itemize}
    
    \item What is the height of the tree?
    \begin{itemize}
        \item The height of the tree is 3
    \end{itemize}
    
    \item List the external nodes
    \begin{itemize}
        \item K, L, F, G, M, I, J (same as the leaves)
    \end{itemize}
    
    \item List the ancestors of G
    \begin{itemize}
        \item The ancestors of G are C and A
    \end{itemize}
    
\end{enumerate}

\section*{2.4 Binary Trees}
\begin{enumerate}
    \item For each of the following trees, determine whether or not the tree is a binary tree, a two-tree, full, and complete and give reasons for your determinations. If there is a subtree that \textbf{does} meet the criteria for the categories above, list the nodes of the subtree.
    \begin{itemize}
        \item i.  
        \item ii. This is a full tree because every node has exactly two or zero children.
    \end{itemize}
    
\end{enumerate}

\section*{2.5 Fix the Code}
\subsection*{The Code}
\begin{lstlisting}
#include <iostream>

using namespace std;

//have this method changed to void, since we are printing within the method
void reverse(string list[], int size) {
    //hard-coded value persists from original code
    string result[6];
    //this for loop replaces each index value of result with the opposite value
    for (int i = 0, j = size - 1; i < size; i++, j--)
        result[j] = list[i];

    //this for loop I wrote iterates through the array, printing each value followed by a space
    for (int j = 0; j < 6; j++)
        cout << result[j] << " ";
}

//changed the value type from int to string
void printArray(string list[], int size) {
    //this code still has its original functionality
    for (int i = 0; i < size; i++)
        cout << list[i] << " ";
}

int main() {
    //changed list array types to string values instead of integers
    //now this code works with not just numbers, but strings and characters too
    //integer values can be parsed out of the array
    string list[] = {"1", "2", "3", "4", "5", "6"};
    printArray(list, 6);
    cout << endl;
    //regular method call instead of assigning value to p
    reverse(list, 6);
    cout << endl;
}
\end{lstlisting}

\subsection{The Output}
\begin{figure}[H]
    \centering
    \includegraphics{Assignmant_2_C}
    \caption{Output for first fixed code}
    \label{fig:Part 2C}
\end{figure}


\end{document}
