\documentclass[11pt]{article}

    \usepackage[utf8]{inputenc}
    \usepackage{listings}
    \usepackage{kotex}
    \usepackage{color}
    \usepackage{float}
    \usepackage{hyperref}
    \usepackage{geometry}
    \geometry{a4paper}
    \geometry{margin = .5in}
    
    \usepackage{graphicx}
    
    \title{Intro to Software Engineering: Assignment 7}
    \author{Harry Haisty}
    
    \begin{document}
    \maketitle
    \section*{Review Questions}
    \begin{enumerate}
    
    \item Explain the role of requirements in architectural design. Explain the role of requirements in detail design. 
    \begin{itemize}
        \item[] The function and nonfunctional requirements along with the technical considerations provide most of the drive for the architecture. 
    \end{itemize}

    \item What does aggregation mean in OO? Give an example.
    \begin{itemize}
        \item[] Aggregates are parts of a larger object. For example, if there is a class called \textit{Car}, then it might be comprised of objects such as \textit{engine} 
        and \textit{wheel}.
    \end{itemize}

    \item When we employ the technique of generalization in design, what are we doing, and which part of OO design is closely related to this concept?
    \begin{itemize}
      \item[] We are making associations between classes based on their commonalitites, and we then identify a common superclass for them. 
      This is most closely related to the OO concept of Logical View.
    \end{itemize}

    \item List two differences between the state transition diagram and the sequence diagram.
    \begin{enumerate}
        \item[] A state chart covers a larger span than the sequence diagram.
        \item[] A sequence diagram is designed to describe one particular function.    
    \end{enumerate}
   
    \item Describe three different views used in architectural design. 
    \begin{enumerate}
        \item[] Logical views -- represents the relationship between classes (Object Oriented View)
        \item[] Process views -- represents the runtime components and how they interact with eachother
        \item[] Subsystem decomposition -- represents the modules and subsystems, joined with export and import relations
    \end{enumerate}

    \item What is the difference between data modeling and logical database design?
    \begin{itemize}
        \item[] Data Modeling is creating an ER model, and Logical Database Design is taking in input as a detailed ER model and and producing a normalized, relational schema.
    \end{itemize}

    \item Describe the difference between low-fidelity and high-fidelity prototyping in the design of the interface. Choose one and give the reasons why you should show the client this prototype. 
    \begin{itemize}
        \item[] I would show a high-fidelity prototype to the consumer, it is a more polished form instead of a hand-drawn design. I would use a low-fidelity model to base my high-fidelity model
        off of. I think it is more professional, if time allows, to present a basic interface so that the consumer has a better idea of what they are buying.
    \end{itemize}

    \item Explain the three columns in Figure 7.26 labeled User, Screen Output, and Process with regard to design. 
    \begin{itemize}
        \item[] \textbf{First column } \textit{First:} this shows the user's interaction with the system.
        \item[] \textbf{Second Column } \textit{Second:} this shows the the two visual outputs, and shows the person's usage of the dropbox function.
        \item[] \textbf{Third Column } \textit{Third:} the inner-workings of the program are shown in the thirs column as a sequence diagram. This column checks to make sure that the students have 
        the necessary prerequisites for the course they are attempting to take. 
        \end{itemize}

    \item Choose one of the cognitive models and explain how the model impacts the design of the user interface. 
    \begin{itemize}
        \item[] Norman developed a model in 1988 that was based on the psychology of everyday actions. He developed a model with 7 different stages. This impacts the design of the system because you
        need to know the \textit{goals} and \textit{expectations} of the users. You should then cater your design to the goals and expectations of the user. 
    \end{itemize}

    \item Visit a website that is from a different country or culture. Give an example of a multicultural 
    issue you found on their site. Explain how you would propose to redesign, taking into consideration the issue found. 
    \begin{itemize}
        \item[] I chose \ https://namu.wiki/w/식사하는\%20철학자\%20문제, which is a Korean version of Wikipedia. It is a very well-organized website, and has a very logical layout. 
        The only issue with the website is that the direct translation of the words is very literal, and some of the higher concepts can be lost in translation. I would work on 
        refining the translation algorithms, so that the higher-level concepts aren't lost when they are translated into a second language.
        
    \end{itemize}
    \end{enumerate}
    
    \end{document}
    