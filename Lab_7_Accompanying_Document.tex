\documentclass[11pt]{article}
\usepackage[utf8]{inputenc}
\usepackage{listings}
\usepackage{color}
\usepackage{graphicx}
\usepackage{float}
\usepackage{geometry}
\geometry{a4paper}
\geometry{margin = .5in}

\usepackage{graphicx}

\definecolor{dkgreen}{rgb}{0,0.6,0}
\definecolor{gray}{rgb}{0.5,0.5,0.5}
\definecolor{mauve}{rgb}{0.58,0,0.82}
\lstset{frame=tb,
  language=C++,
  aboveskip=3mm,
  belowskip=3mm,
  showstringspaces=false,
  columns=flexible,
  basicstyle={\small\ttfamily},
  numbers=none,
  numberstyle=\tiny\color{gray},
  keywordstyle=\color{mauve},
  commentstyle=\color{dkgreen},
  stringstyle=\color{blue},
  breaklines=true,
  breakatwhitespace=true,
  tabsize=3
}

\title{Lab 7  Accompanying Document}
\author{Harry Haisty}
\date{September 2018}

\begin{document}

\maketitle

\section*{The Code}
\begin{lstlisting}
#include <iostream>
using namespace std;

template <typename GenericType>
GenericType maxValue(const GenericType& value1, const GenericType& value2)
{
    if(value1 > value2)
        return value1;
    else
        return value2;
}

double maxValue(const double& value1, const double& value2)
{
    if (value1 > value2)
        return value1;
    else
        return value2;
}

int maxValue(const int& value1, const int& value2){
    if (value1 > value2)
        return  value1;
    else
        return value2;
}

int main() {
    int i = 5, j = 6, k, z, y;
    long l = 10, m = 5, n;
    k = maxValue<double>(i, j);
    z = maxValue(i, j);
    y = maxValue<int>(i, j);
    cout << k << endl;
    cout << z << endl;
    cout << y << endl;
}
\end{lstlisting}

\section*{The Explanation}
The object of this assignment was to explore different ways for a programmer to implement and replace overload functions. Using a template of a generic type, I was able to limit my code to only a few lines where before there had to be different methods to account for each different data type. 
The generic type template was able to take in a doubl, and int, or another value, and compare those two values taken in to see which one of the values was larger. 

\section*{The Output}
\begin{figure}[h]
\centering
\includegraphics[width = 15cm]{lab_7_1}
\caption {output for part 1}
\end{figure}

\end{document}
