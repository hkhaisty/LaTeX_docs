\documentclass[11pt]{article}
\usepackage[utf8]{inputenc}
\usepackage{listings}
\usepackage{color}
\usepackage{float}
\usepackage{geometry}
\geometry{a4paper}
\geometry{margin = .5in}

\usepackage{graphicx}

\definecolor{dkgreen}{rgb}{0,0.6,0}
\definecolor{gray}{rgb}{0.5,0.5,0.5}
\definecolor{mauve}{rgb}{0.58,0,0.82}
\lstset{
  language=C++,
  aboveskip=3mm,
  belowskip=3mm,
  showstringspaces=false,
  columns=flexible,
  basicstyle={\small\ttfamily},
  numbers=none,
  numberstyle=\tiny\color{gray},
  keywordstyle=\color{mauve},
  commentstyle=\color{dkgreen},
  stringstyle=\color{blue},
  breaklines=true,
  breakatwhitespace=true,
  tabsize=3
}


\title{Lab 11 Accompanying Document}
\author{Harry Haisty}
\date{}

\begin{document}

\maketitle

This particular assignment I completed to the best of my ability. I used my understanding of stacks and heaps, but I was not able to make it all the way through the entire lab. I used CLion to write this code, but I didn't realize until much later that I had to manually change the version of Cmake so that I didn't have to keep deleting and restarting my project. I realize that this could be solved by writing the code in a solution file, but instead of doing that I wanted to keep it a regular .cpp file so that you could read it. 

\section*{Introduction}

\subsection*{Section 11B}
\begin{lstlisting}
#include <iostream>
#include <vector>
#include <algorithm>

using namespace std;

void print_methods(vector<int> vector_1, int user_int) {
    //call built-in make_heap function using beginning and end of vector
    make_heap(vector_1.begin(), vector_1.end());
    //call front value of heap and print
    cout << "\n1. First max heap: " << vector_1.front() << "\n";

    //pop front value of heap to get rid of old max value
    pop_heap(vector_1.begin(), vector_1.end());
    vector_1.pop_back();
    cout << "\n2. Second max heap (after pop): " << vector_1.front() << "\n";

    //prompt user to enter value to push onto heap
    cout << "\n3. Enter a value to push onto heap: ";
    cin >> user_int;
    vector_1.push_back(user_int);
    push_heap(vector_1.begin(), vector_1.end());

    //print out new value after last push
    cout << "\n4. New max heap after push: " << vector_1.front() << "\n";

    //call sorting function for heap
    sort_heap(vector_1.begin(), vector_1.end());
    cout << "\n5. Final sorted heap: ";

    //prints formatted list of sorted heap
    for (unsigned i = 0; i < vector_1.size(); i++)
        cout << ' ' << vector_1[i];

    cout << '\n';
}

int main() {
    int user_integer;
    int array_size;

    //accepts user input to make array size
    cout << "Please indicate size of array: ";
    cin >> array_size;

    //declares and initializes array with user-defined size
    int my_ints[array_size];

    //loops through adding user values to array
    for (int i = 0; i <= array_size; i++) {

        //prompts user to enter integer value to populate array
        cout << "Please enter integer value at place " << i << ":";
        //accepts user integer
        cin >> user_integer;
        //places input at index in array
        my_ints[i] = user_integer;
    }

    //declare and initialize int vector
    std::vector<int> vector(my_ints, my_ints + 5);

    //call print methods logic
    print_methods(vector, user_integer);
}

\end{lstlisting}

\end{document}
