\documentclass{article}
\usepackage{geometry}
\usepackage[utf8]{inputenc}

\usepackage{graphicx}

\geometry{margin=.5in}

\title{Lab 12 Accompanying Document}
\author{Harry Haisty}
\date{}

\begin{document}

\maketitle

\section*{Introduction}

\subsection*{Part A: Getting the code to run}

This part of the lab was mainly about separating the code into a couple of different header files and making sure that they included the proper elements. Once the code that is supposed to go into the header files goes into the header files, they no longer have access to the  $<$iostream$>$ library for input and output, thus rendering some of the cout and cin functions useless.

\begin{figure}[h]
    \centering
    \includegraphics[width=10cm]{12part1}
   % \caption{Caption}
    \label{fig:my_label}
\end{figure}

\begin{figure}[h]
    \centering
    \includegraphics[width=10cm]{12part2}
   % \caption{Caption}
    \label{fig:my_label}
\end{figure}

\subsection*{Part B: The summary report}

The graph class cannot exist without the declaration of a generic template ``T". Different uses of the template are then implemented, some in which three variables are passed in, of mixed type int and vector. 


\section{The Code}
\begin{lstlisting}
#include <vector>

using namespace std;

#include "Edge.h"
#include "Tree.h"
#include "graph.h"
#include <vector>
#include <queue>

using namespace std;

template<typename T>
Graph<T>::Graph() {
}

template<typename T>
Graph<T>::Graph(vector<T> vertices, int edges[][2],
                int numberOfEdges) {
    this->vertices = vertices;
    createAdjacencyLists(vertices.size(), edges, numberOfEdges);
}

template<typename T>
Graph<T>::Graph(int numberOfVertices, int edges[][2],
                int numberOfEdges) {
    for (int i = 0; i < numberOfVertices; i++)
        vertices.push_back(i); // vertices is {0, 1, 2, ..., n-1}
    createAdjacencyLists(numberOfVertices, edges, numberOfEdges);
}

template<typename T>
Graph<T>::Graph(vector<T> vertices, vector<Edge> edges) {
    this->vertices = vertices;
    createAdjacencyLists(vertices.size(), edges);
}

template<typename T>
Graph<T>::Graph(int numberOfVertices, vector<Edge> edges) {
    for (int i = 0; i < numberOfVertices; i++)
        vertices.push_back(i); // vertices is {0, 1, 2, ..., n-1}
    createAdjacencyLists(numberOfVertices, edges);
}

template<typename T>
void Graph<T>::createAdjacencyLists(int numberOfVertices,
                                    int edges[][2], int numberOfEdges) {
    for (int i = 0; i < numberOfVertices; i++) {
        neighbors.push_back(vector<int>(0));
    }
    for (int i = 0; i < numberOfEdges; i++) {
        int u = edges[i][0];
        int v = edges[i][1];
        neighbors[u].push_back(v);
    }
}

template<typename T>
void Graph<T>::createAdjacencyLists(int numberOfVertices,
                                    vector<Edge> &edges) {
    for (int i = 0; i < numberOfVertices; i++) {
        neighbors.push_back(vector<int>(0));
    }
    for (int i = 0; i < edges.size(); i++) {
        int u = edges[i].u;
        int v = edges[i].v;
        neighbors[u].push_back(v);
    }
}

template<typename T>
int Graph<T>::getSize() const {
    return vertices.size();
}

template<typename T>
int Graph<T>::getDegree(int v) const {
    return neighbors[v].size();
}

template<typename T>
T Graph<T>::getVertex(int index) const {
    return vertices[index];
}

template<typename T>
int Graph<T>::getIndex(T v) const {
    for (int i = 0; i < vertices.size(); i++) {
        if (vertices[i] == v)
            return i;
    }
    return -1; // If vertex is not in the graph
}

template<typename T>
vector<T> Graph<T>::getVertices() const {
    return vertices;
}

template<typename T>
vector<int> Graph<T>::getNeighbors(int v) const {
    return neighbors[v];
}

template<typename T>
void Graph<T>::printEdges() const {
    for (int u = 0; u < neighbors.size(); u++) {
        cout << "Vertex " << u << ": ";
        for (int j = 0; j < neighbors[u].size(); j++) {
            cout << "(" << u << ", " << neighbors[u][j] << ") ";
        }
        cout << endl;
    }
}

template<typename T>
void Graph<T>::printAdjacencyMatrix() const {
    int size = vertices.size();
    // Use vector for 2-D array
    vector<vector<int>> adjacencyMatrix(size);
    // Initialize 2-D array for adjacency matrix
    for (int i = 0; i < size; i++) {
        adjacencyMatrix[i] = vector<int>(size);
    }
    for (int i = 0; i < neighbors.size(); i++) {
        for (int j = 0; j < neighbors[i].size(); j++) {
            int v = neighbors[i][j];
            adjacencyMatrix[i][v] = 1;
        }
    }
    for (int i = 0; i < adjacencyMatrix.size(); i++) {
        for (int j = 0; j < adjacencyMatrix[i].size(); j++) {
            cout << adjacencyMatrix[i][j] << " ";
        }
        cout << endl;
    }
}

template<typename T>
Tree Graph<T>::dfs(int v) const {
    vector<int> searchOrders;
    vector<int> parent(vertices.size());
    for (int i = 0; i < vertices.size(); i++)
        parent[i] = -1; // Initialize parent[i] to -1
    // Mark visited vertices
    vector<bool> isVisited(vertices.size());
    for (int i = 0; i < vertices.size(); i++) {
        isVisited[i] = false;
    }
    // Recursively search
    dfs(v, parent, searchOrders, isVisited);
    // Return a search tree
    return Tree(v, parent, searchOrders);
}

template<typename T>
void Graph<T>::dfs(int v, vector<int> &parent,
                   vector<int> &searchOrders, vector<bool> &isVisited) const {
    // Store the visited vertex
    searchOrders.push_back(v);
    isVisited[v] = true; // Vertex v visited
    for (int j = 0; j < neighbors[v].size(); j++) {
        int i = neighbors[v][j];
        if (!isVisited[i]) {
            parent[i] = v; // The parent of vertex i is v
            dfs(i, parent, searchOrders, isVisited); // Recursive search
        }
    }
}

template<typename T>
Tree Graph<T>::bfs(int v) const {
    vector<int> searchOrders;
    vector<int> parent(vertices.size());
    for (int i = 0; i < vertices.size(); i++)
        parent[i] = -1; // Initialize parent[i] to -1
    queue<int> queue; // list used as a queue
    vector<bool> isVisited(vertices.size());
    queue.push(v); // Enqueue v
    isVisited[v] = true; // Mark it visited
    while (!queue.empty()) {
        int u = queue.front(); // Get the top element
        queue.pop(); // remove the top element
        searchOrders.push_back(u); // u searched
        for (int j = 0; j < neighbors[u].size(); j++) {
            int w = neighbors[u][j];
            if (!isVisited[w]) {
                queue.push(w); // Enqueue w
                parent[w] = u; // The parent of w is u
                isVisited[w] = true; // Mark it visited
            }
        }
    }
    return Tree(v, parent, searchOrders);
}
///////////////
#include <iostream>
#include <string>
#include <vector>
#include "Graph.h"
#include "Edge.h"

using namespace std;

int main() {
    // Vertices for first graph
    string vertices[] = {"Seattle", "San Francisco", "Los Angeles",
                         "Denver", "Kansas City", "Chicago", "Boston", "New York",
                         "Atlanta", "Miami", "Dallas", "Houston"};
    // Edge array for first graph
    int edges[][2] = {
            {0,  1},
            {0,  3},
            {0,  5},
            {1,  0},
            {1,  2},
            {1,  3},
            {2,  1},
            {2,  3},
            {2,  4},
            {2,  10},
            {3,  0},
            {3,  1},
            {3,  2},
            {3,  4},
            {3,  5},
            {4,  2},
            {4,  3},
            {4,  5},
            {4,  7},
            {4,  8},
            {4,  10},
            {5,  0},
            {5,  3},
            {5,  4},
            {5,  6},
            {5,  7},
            {6,  5},
            {6,  7},
            {7,  4},
            {7,  5},
            {7,  6},
            {7,  8},
            {8,  4},
            {8,  7},
            {8,  9},
            {8,  10},
            {8,  11},
            {9,  8},
            {9,  11},
            {10, 2},
            {10, 4},
            {10, 8},
            {10, 11},
            {11, 8},
            {11, 9},
            {11, 10}
    };
    const int NUMBER_OF_EDGES = 46; //
    // Create vector for vertices
    vector<string> vectorOfVertices(12);
    copy(vertices, vertices + 12, vectorOfVertices.begin());
    Graph<string> graph1(vectorOfVertices, edges, NUMBER_OF_EDGES);
    cout << "The number of vertices in graph1: " << graph1.getSize();
    cout << "\nThe vertex with index 1 is " << graph1.getVertex(1);
    cout << "\nThe index for Miami is " << graph1.getIndex("Miami");
    cout << "\nedges for graph1: " << endl;
    graph1.printEdges();
    cout << "\nAdjacency matrix for graph1: " << endl;
    graph1.printAdjacencyMatrix();
    // vector of Edge objects for second graph
    vector<Edge> edgeVector;
    edgeVector.push_back(Edge(0, 2));
    edgeVector.push_back(Edge(1, 2));
    edgeVector.push_back(Edge(2, 4));
    edgeVector.push_back(Edge(3, 4));
    // Create a graph with 5 vertices
    Graph<int> graph2(5, edgeVector);
    cout << "The number of vertices in graph2: " << graph2.getSize();
    cout << "\nedges for graph2: " << endl;
    graph2.printEdges();
    cout << "\nAdjacency matrix for graph2: " << endl;
    graph2.printAdjacencyMatrix();
    Tree dfs = graph1.dfs(5); // Vertex 5 is Chicago
    vector<int> searchOrders = dfs.getSearchOrders();
    cout << dfs.getNumberOfVerticesFound() <<
         " vertices are searched in this DFS order:" << endl;
    for (int i = 0; i < searchOrders.size(); i++)
        cout << graph1.getVertex(searchOrders[i]) << " ";
    cout << endl << endl;
    for (int i = 0; i < searchOrders.size(); i++)
        if (dfs.getParent(i) != -1)
            cout << "parent of " << graph1.getVertex(i) <<
                 " is " << graph1.getVertex(dfs.getParent(i)) << endl;
    return 0;
}
////////////////
\end{lstlisting}


\end{document}
