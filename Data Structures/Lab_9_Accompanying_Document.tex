\documentclass[11pt]{article}
\usepackage[utf8]{inputenc}
\usepackage{listings}
\usepackage{color}
\usepackage{float}
\usepackage{geometry}
\geometry{a4paper}
\geometry{margin = .5in}

\usepackage{graphicx}

\definecolor{dkgreen}{rgb}{0,0.6,0}
\definecolor{gray}{rgb}{0.5,0.5,0.5}
\definecolor{mauve}{rgb}{0.58,0,0.82}
\lstset{
  language=C++,
  aboveskip=3mm,
  belowskip=3mm,
  showstringspaces=false,
  columns=flexible,
  basicstyle={\small\ttfamily},
  numbers=none,
  numberstyle=\tiny\color{gray},
  keywordstyle=\color{mauve},
  commentstyle=\color{dkgreen},
  stringstyle=\color{blue},
  breaklines=true,
  breakatwhitespace=true,
  tabsize=3
}

\title{Lab 9 Accompanying Document}
\date{}
\author{Harry Haisty}
\begin{document}

\maketitle

\section*{Introduction}
I improved the user interface by making clean, accessible options for the user. I clearly lableled the options that the users had for train car selection, and displayed the input necessary for the users to actually add or remove those particular pieces of the train. I then implemented functions to add and remove trains from the car, either in the front or back of the train. 

\begin{figure}[h]
    \centering
    \includegraphics[width = 10cm]{9a}
    \caption{Output for train interface }
\end{figure}

\begin{figure}[h]
    \centering
    \includegraphics[width = 10cm]{9b}
    \caption{second part of output}
\end{figure}

\newpage \newpage
\section*{} 
\newpage
\subsection*{declarations.h}
\begin{lstlisting}
using namespace std;

class container {
public:
    container() {}
};

class box : public container {
public:

    box() {}
};

class rail_car {
public:
    rail_car() {}
};

class tank_car : public rail_car {
public:
    tank_car() {}
};

class box_car : public rail_car, public box {
public:
    box_car() {
    }
};

class engine : public rail_car {
public:
    engine() {}
};

class caboose : public rail_car {
public:
    caboose() {}
};


\end{lstlisting}



\newpage
\subsection*{main.cpp}
\begin{lstlisting}
#include <iostream>
#include "declarations.h"

const int MAX_SIZE = 100;
rail_car *train[MAX_SIZE];

int car_count;

int interpret_ar(string input);

int interpret_trains(string input);

int interpret_fb(string input);

bool add_to_front_of_train();

void take_off_front_of_train();

bool add_to_back_of_train();

void take_off_back_of_train();

void shift_array_to_right();

void shift_array_to_left();


int main() {
    string input;
    car_count = 0;
    bool loop;
    do {
        if (car_count == 0) {
            loop = add_to_back_of_train();
            cout << "There are " << car_count << " cars in the array."
                 << endl;
        } else {
            cout << "Type 'add' to add a car, 'remove' to remove a car, or 'exit' to exit" << endl;
            cin >> input;
            switch (interpret_ar(input)) {
                //this removes car from array
                case 1:
                    //if there is only one car, this removes it from the train
                    if (car_count == 1) {
                        //function call to remove car
                        take_off_back_of_train();
                        //prints out result after function call
                        cout << "There are " << car_count << " cars in the array." << endl;
                    } else {
                        //if there is more than one car in the array, prompt user to pick front or back
                        cout << "Would you like to remove a car at the front or back? (e to exit)" << endl;
                        //accept user input
                        cin >> input;
                        //switch statement to determine whether front or back should be taken off
                        switch (interpret_fb(input)) { //option menu
                            case 0:
                                //removes car from front of train
                                take_off_front_of_train();
                                //prints result of removing car
                                cout << "There are " << car_count << " cars in the array." << endl;
                                //break so we don't continue running through cases
                                break;
                            case 1:
                                //removes car from back of train
                                take_off_back_of_train();
                                //prints result of removing car
                                cout << "There are " << car_count << " cars in the array." << endl;
                                //break
                                break;
                            case -1:
                                //breaks if input is invalid
                                loop = false;
                                break;
                            default:
                                cout << "Invalid input. " << endl;
                        }
                        break;
                        case 0:
                            //prompt user to add trains to front
                            cout << "type 'front' to add a car to the front of your train "
                                    "\ntype 'back' to add a car to the back of the train"
                                    "\ntype 'exit' to exit" << endl;
                        cin >> input;
                        switch (interpret_fb(input)) { //option menu
                            case 0:
                                add_to_front_of_train();
                                cout << "There are " << car_count << " cars in the array." << endl;
                                break;
                            case 1:
                                add_to_back_of_train();
                                cout << "There are " << car_count << " cars in the array." << endl;
                                break;
                            case -1:
                                loop = false;
                                break; //exit
                            default:
                                cout << "Invalid input. " << endl;
                        }
                        break;
                        case -1:
                            loop = false;
                        break; //exit
                        default:
                            cout << "Invalid input. " << endl;
                    }
            }
        }
    } while (loop); //loops till exit requested
    // ??
    cout << "There are " << car_count << " cars in the array." << endl; //outputs final number of cars

    return 0; //added this
}

//implementation of functions
int interpret_ar(string input) { //interprets input for add/remove
    if (input == "add") return 0;
    else if (input == "remove") return 1;
    else if (input == "exit") return -1;
    else return 2;
}

int interpret_trains(string input) { //interprets input for car picking
    if (input == "engine") return 0;
    else if (input == "box") return 1;
    else if (input == "tank") return 2;
    else if (input == "caboose") return 3;
    else if (input == "exit") return -1;
    else if (input == "help") return -2;
    else return 4;
}

int interpret_fb(string input) { //interprets input for front/back
    if (input == "front") return 0;
    else if (input == "back") return 1;
    else if (input == "exit") return -1;
    else return 2;
}

bool add_to_back_of_train() { //adds car to back
    cout << "What type of rail car would you like to add?"
            "\ntype 'engine' to add an engine car"
            "\ntype 'box' to add a box car"
            "\ntype 'caboose' to add a caboose car"
            "\ntype 'tank' to add a tank car\n"
            "('exit' to exit, 'help' for help)" << endl;
    string input;
    cin >> input;
    switch (interpret_trains(input)) {
        case 0:
            train[car_count++] = new engine;
            break;
        case 1:
            train[car_count++] = new box_car;
            break;
        case 2:
            train[car_count++] = new tank_car;
            break;
        case 3:
            train[car_count++] = new caboose;
            break;
        case -1:
            return false;
        case -2:
            cout << "Options: engine, box, tank, caboose" << endl;
            break;
        default:
            cout << "Invalid input. " << endl;
            break;
    }
    return true;
}

bool add_to_front_of_train() { //adds car to back
    cout << "What type of rail car would you like to add?"
            "\ntype 'Engine' to add an engine car"
            "\ntype 'Box' to add a box car"
            "\ntype 'Caboose' to add a caboose car"
            "\ntype 'Tank' to add a tank car"
            "\n('exit' to exit, 'help' for help)" << endl;
    string input;
    cin >> input;
    switch (interpret_trains(input)) {
        case 0:
            shift_array_to_right();
            train[0] = new engine;
            car_count++;
            break;
        case 1:
            shift_array_to_right();
            train[0] = new box_car;
            car_count++;
            break;
        case 2:
            shift_array_to_right();
            train[0] = new tank_car;
            car_count++;
            break;
        case 3:
            shift_array_to_right();
            train[0] = new caboose;
            car_count++;
            break;
        case -1:
            return false;
        case -2:
            cout << "Options: engine, box, tank, caboose" << endl;
            break;
        default:
            cout << "Invalid input. " << endl;
            break;
    }
    return true;
}

void take_off_front_of_train() { //removes car from front
    delete train[0];
    shift_array_to_left;
    --car_count;
}

void take_off_back_of_train() { //removes car from back
    delete train[--car_count];
}

void shift_array_to_right() { //shifts array right for removal
    if (car_count == MAX_SIZE) cout << "Maximum car limit reached." << endl;
    else {
        for (int i = car_count - 1; i >= 0; i--) {
            train[i + 1] = train[i];
        }
    }
}

void shift_array_to_left() { //shift array left for removal
    for (int i = 0; i < car_count - 1; i++) {
        train[i] = train[i + 1];
    }
}
\end{lstlisting}

\end{document}
