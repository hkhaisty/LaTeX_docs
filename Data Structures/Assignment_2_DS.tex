\documentclass[11pt]{article}

\usepackage[utf8]{inputenc}
\usepackage{listings}
\usepackage{float}
\usepackage{xcolor}
\usepackage{geometry}

\geometry{a4paper}
\geometry{margin = .5in}

\usepackage{graphicx}

\definecolor{dkgreen}{rgb}{0,0.6,0}
\definecolor{gray}{rgb}{0.5,0.5,0.5}
\definecolor{mauve}{rgb}{0.58,0,0.82}
\lstset{frame=tb,
  language=C++,
  aboveskip=3mm,
  belowskip=3mm,
  showstringspaces=false,
  columns=flexible,
  basicstyle={\small\ttfamily},
  numberstyle=\tiny\color{gray},
  numbers=left, 
  keywordstyle=\color{orange},
  breaklines=true,
  breakatwhitespace=true,
  frame=tb, 
  tabsize=4, 
  commentstyle=\color{gray}, 
  stringstyle=\color{dkgreen} 
}

\title{Data Structures: Assignment 2}
\author{Harry Haisty}

\begin{document}
\maketitle
\centerline{Codename: Cheese Puff}

\subsection*{Computing Factorials with Recursive Functions}
\begin{lstlisting}
#include <iostream>

using namespace std;

//make the type long long to accomodate the large factorial values
long long factorial(long long n) {
    //the factorial value only goes to 1, if it were multiplied by 0 then the whole value would be 0
    if (n > 1)
        //recursively decrement the value of n and multiply
        return n * factorial(n - 1);
    else
        //if the input value is 1 or less, return just the value "1"
        return 1;
}

int main() {
    //declare user input as type long long
    long long user_input;

    //prompt user to input value
    cout << "Enter an integer: ";
    //take in user input value
    cin >> user_input;

    //print out recursive call to factorial method
    cout << factorial(user_input) << endl;
}
\end{lstlisting}
\subsection*{The Output}
\begin{figure}[H]
    \centering
    \includegraphics{assignment_2_1}
    \caption{Output for factorial recursion program}
    \label{fig:part 1}
\end{figure}

\subsection{The Description}
I made the data types long long because factorials grow to such large sizes that I kept running into trouble, even with regular long. The way that I calculate the factorial is the same way we learned to in class, with a recursive function call to the factorial method, all the while decrementing the arbitrary value "n" so that you are continually getting closer to 1. If you include 0, then your entire value will be 0, so it has to be to 1. I checked the value of my output against the scientific calculator on my computer.

\subsection*{Computing Sums with Recursive Functions}
\begin{lstlisting}
#include <iostream>

using namespace std;

//works similarly to the factorial method
int sum(int n) {
    //n must be above 0, otherwise the loop will continue negatively forever
    if (n > 0)
        return n + sum(n - 1);

        //if n is less than or equal to 0, return the value 0.
        // if you return 1, then 1 will be added to the final sum value
    else
        return 0;
}

int main() {
    //declare user input
    int user_input = 0;

    //prompt user input
    cout << "Enter an integer: ";
    //accept user input and initialize user input value
    cin >> user_input;

    //call recursive function with user input value
    cout << sum(user_input);
}
\end{lstlisting}

\subsection*{The Output}
\begin{figure}[H]
    \centering
    \includegraphics{assignment_2_2}
    \caption{Output for sum of integers program}
    \label{fig:part 2}
\end{figure}

\subsection*{The Description}
This piece of code works similarly to the factorial method, in that it calls itself and it continuously decrementing until it reaches a small value. In this case, the value is 0. The value is 0 because we are calculating the sum, with addition, and will not invalidate our answers by including 0 in the operation. This method also needs to have a base case, otherwise it will invoke infinite recursion. 


\subsection*{Calculating Fibonacci Sequence with Recursive Function}
\begin{lstlisting}
#include <iostream>

using namespace std;

int fibbonacci_sequence(int n) {

    //declares and initializes num1 and num2 values to 0 and 1 so that program doesn't hang
    int num1 = 0, num2 = 1;

    //stops loop if n is smaller than threshold
    if (n < 1)
        return n;

    //loops through to user input value
    for (int i = 1; i < n; i++) {
        //prints value
        printf("%d", num2, " ");
        //sets next to sum of first two integers, per the fibonacci sequence
        int next = num1 + num2;
        //sets smaller num1 to larger num2
        num1 = num2;
        //sets num2 to larger number
        num2 = next;
    }
}

int main() {

    //declares integer N
    int N;

    //prompts user to set integer N to value
    cout << "Enter an integer value: ";
    //sets integer N to user value
    cin >> N;

    //prints out call to fibonacci sequence
    cout << fibbonacci_sequence(N);
}
\end{lstlisting}

\subsection*{The Output}
\begin{figure}[H]
    \centering
    \includegraphics{assignment_2_3}
    \caption{Output for Fibonacci sequence up to a value}
    \label{fig:part 3}
\end{figure}

\subsection*{The Description}
This piece of code works a little differently than the last two because it requires that two separate values be incremented. It also requires that you hard-code the first couple of values, otherwise you will go out of bounds and never actually be able to calculate the \textit{nth} Fibonacci value.


\subsection*{Evaluating Palindromes using Recursion}
\begin{lstlisting}
#include <iostream>

using namespace std;

//return boolean value to user to show if palindrome value is true or not
bool is_palindrome(const string &str, int first_value, int last_value) {
    //if first and last values left are equal, return true
    if (first_value >= last_value)
        return true;
    //if the first and last values left are not equal, return false
    if (str[first_value] != str[last_value])
        return false;
    //increment first value and decrement last value to evaluate against eachother
    return is_palindrome(str, ++first_value, --last_value);
}

int main() {
    //declare user input value
    string user_string;

    //prompt user to enter a string
    cout << "enter a string: ";
    //initialize user input value
    cin >> user_string;

    //print output based on returned value from recursive function
    if (is_palindrome(user_string, 0, user_string.length() - 1) == 1)
        cout << user_string << " is a palindrome.";
    else
        cout << user_string << " is not a palindrome";
}
\end{lstlisting}

\subsection*{The Output}
\begin{figure}[H]
    \centering
    \includegraphics{assignment_2_4}
    \caption{Output for evaluating truth of palindrome}
    \label{fig:part 4}
\end{figure}

\subsection*{The Description}
This piece of code begins by evaluating a string at its first and last indices. If the first and last indices are the same, then the first value increments and the last value decrements. If there any sort of disparity between the two values being evaluated, the code exits and the value is returned as false. If they are evaluated to be equal, then the value is returned as true. I wrote an if and an else statement so that the output wasn't just \textit{1} and \textit{0}. Now the user actually knows that \textit{their} word is or isn't a palindrome.

\subsection*{Using Recursion to Print Strings Backwards}
\begin{lstlisting}
#include <iostream>

using namespace std;

string reverse_string(string input, string reverse, int n) {
    //base case, return empty string
    if (n == 0)
        return "";

    //reverse elements, append individual elements to new string
    return reverse += input.back() + reverse_string(input.substr(0, n - 1), reverse, n - 1);
}

int main() {
    //declare user input
    string user_input;

    //prompt user for input
    cout << "Enter a string: ";
    //accept user input
    cin >> user_input;

    //print out the reverse of the string by calling recursive function
    cout << reverse_string(user_input, "", user_input.length()) << endl;
}
\end{lstlisting}

\subsection*{The Output}
\begin{figure}[H]
    \centering
    \includegraphics{assignment_2_5}
    \caption{Output for printing strings backwards}
    \label{fig:part 5}
\end{figure}

\subsection*{The Description}
This piece of code reverses the elements, sets "reverse" equal to a value, and then continually appends the reverse elements to the end of the string.


\subsection*{Calculating Iterative Time vs. Recursive Time}
\begin{lstlisting}
#include <iostream>
#include <chrono>

using namespace std;

long sum;

long iterative_sum(int n) {
    for (int i = 0; i < n; ++i)
        sum += i;
}

long recursive_sum(int n) {
    if (n > 0)
        n + recursive_sum(n - 1);
}

int main() {
    int user_input = 0;
    typedef std::ratio<1l, 1000000000000l> pico;

    cout << "Enter an integer: ";
    cin >> user_input;

    auto start = chrono::steady_clock::now();
    iterative_sum(user_input);
    auto end = chrono::steady_clock::now();

    chrono::duration<double, std::pico> elapsed_time = (end - start);
    cout << elapsed_time.count() << " picoseconds" << endl;

    start = chrono::steady_clock::now();
    recursive_sum(user_input);
    end = chrono::steady_clock::now();

    chrono::duration<double, std::pico> elapsed_time_2 = (end - start);
    cout << elapsed_time_2.count() << " picoseconds";
}

\end{lstlisting}

\subsection*{The Output}
\begin{figure}[H]
    \centering
    \includegraphics{assignment_2_B}
    \caption{Output for calculating difference in runtime}
    \label{fig:part 6}
\end{figure}

\subsection*{The Description}
This one I couldn't figure out the whole way. Adding larger and larger numbers didn't affect the runtime, and I even created a custom ratio value so that I could get smaller and smaller increments of time. I ended up whittling it down to a picosecond, but the result still didn't show any difference.


\section*{2.3 Trees}
\begin{enumerate}
    \item List the leaves of the tree
    \begin{itemize}
        \item K, L, F, G, M, I, J
    \end{itemize}
    
    \item List the internal nodes of the tree
    \begin{itemize}
        \item A, B, C, D, E, H
    \end{itemize}
    
    \item List a subtree
    \begin{itemize}
        \item B, E, K, L
    \end{itemize}
    
    \item Name a child and its parent node
    \begin{itemize}
        \item C: \textit{parent node}, F: \textit{child node}
    \end{itemize}
    
    \item What is a branch?
    \begin{itemize}
        \item  a node that has child nodes
    \end{itemize}
    
    \item Name sibling nodes
    \begin{itemize}
        \item C and D are sibling nodes because they come from the same parent node
    \end{itemize}
    
    \item List all descendants of B
    \begin{itemize}
        \item E, K, L
    \end{itemize}
    
    \item What is the path from A to J?
    \begin{itemize}
        \item A, D, J
    \end{itemize}
    
    \item What is the root node?
    \begin{itemize}
        \item A is the root node
    \end{itemize}
    
    \item What is the depth of H?
    \begin{itemize}
        \item The depth of H is 2
    \end{itemize}
    
    \item What is the height of the tree?
    \begin{itemize}
        \item The height of the tree is 3
    \end{itemize}
    
    \item List the external nodes
    \begin{itemize}
        \item K, L, F, G, M, I, J (same as the leaves)
    \end{itemize}
    
    \item List the ancestors of G
    \begin{itemize}
        \item The ancestors of G are C and A
    \end{itemize}
    
\end{enumerate}

\section*{2.4 Binary Trees}

\begin{enumerate}
    \item[a.] For each of the following trees, determine whether or not the tree is a binary tree, a two-tree, full, and complete and give reasons for your determinations. If there is a subtree that \textbf{does} meet the criteria for the categories above, list the nodes of the subtree.
    \begin{itemize}
        \item i.  This tree is neither complete not full, there are nodes that only have one child, and the nodes are not as far left as possible.
        \item ii. This is a full tree because every node has exactly two or zero children.
        \item iii. This tree is neither complete nor full, because many of the nodes have only one child and the nodes are not as far lest as possible.
        \item iv. This is a complete tree because the nodes on the far left are completely full 
    \end{itemize}
    \item[b.] For each tree state: the height, the number of nodes, the number of external nodes, and the number of internal nodes. 
    \begin{enumerate}
        \item[i. ] \textit{Height:} 3, \textit{\# of Nodes:} 11, \textit{\# of External Nodes:} 5, \textit{\# of Internal Nodes:} 6 
        \item[ii. ] \textit{Height:} 3, \textit{\# of Nodes:} 9, \textit{\# of External Nodes:} 5, \textit{\# of Internal Nodes:} 4
        \item[iii. ] \textit{Height:} 4, \textit{\# of Nodes:} 18, \textit{\# of External Nodes:} 8, \textit{\# of Internal Nodes:} 10
        \item[iv. ] \textit{Height:} 4, \textit{\# of Nodes:} 14, \textit{\# of External Nodes:} 7, \textit{\# of Internal Nodes:} 7
    \end{enumerate}
     
    \item[c.] For trees I and II perform a traversal of each of the following types: inorder, postorder, preorder, and level order.
    \begin{enumerate}
        \item[i. ] \textit{inorder:} 5, 4, 6, 10, 8, 11, 17, 19, 31, 43, 49, 11. \textit{postorder:}, 5, 10, 8, 4, 17, 31, 49, 43, 6, 19, 11. \textit{preorder:} 11, 6, 4, 5, 8, 10, 19, 17, 43, 31, 49. \textit{level order:} 11, 6, 19, 4, 8, 17, 43, 5, 10, 31, 49.
        
        \item[ii. ] \textit{inorder:} a, -, b, /, c, +, d, *, e. \textit{postorder:} e, d, *, c, b, /, a, -, +. \textit{preorder:} +, -, a, /, b, c, *, d, e.  \textit{level order:} +, -, *, a, /, d, e, b, c.
    \end{enumerate}
    
    
    
\end{enumerate}

\section*{2.5 Fix the Code}
\subsection*{The Code}
\begin{lstlisting}
#include <iostream>

using namespace std;

//have this method changed to void, since we are printing within the method
void reverse(string list[], int size) {
    //hard-coded value persists from original code
    string result[6];
    //this for loop replaces each index value of result with the opposite value
    for (int i = 0, j = size - 1; i < size; i++, j--)
        result[j] = list[i];

    //this for loop I wrote iterates through the array, printing each value followed by a space
    for (int j = 0; j < 6; j++)
        cout << result[j] << " ";
}

//changed the value type from int to string
void printArray(string list[], int size) {
    //this code still has its original functionality
    for (int i = 0; i < size; i++)
        cout << list[i] << " ";
}

int main() {
    //changed list array types to string values instead of integers
    //now this code works with not just numbers, but strings and characters too
    //integer values can be parsed out of the array
    string list[] = {"1", "2", "3", "4", "5", "6"};
    printArray(list, 6);
    cout << endl;
    //regular method call instead of assigning value to p
    reverse(list, 6);
    cout << endl;
}
\end{lstlisting}

\subsection{The Output}
\begin{figure}[H]
    \centering
    \includegraphics{Assignmant_2_C}
    \caption{Output for first fixed code}
    \label{fig:Part 2C}
\end{figure}

\subsection*{The Description}
I changed the input type for this piece of code to strings and string arrays. This way I could potentially put an array of letters, symbols, or numbers in. The printArray method was working as expected, I just had to change the data type to match the string array. The first for loop in the original code worked as expected, it just wasn't returning the correct value. Instead, I decided to cout the output by iterating through a for loop that prints the individual indices of the string array. 

\subsection*{}
\begin{lstlisting}
#include <iostream>
using namespace std;
//this code works as intended, so I made no modifications
int* reverse(const int* list, int size)
{
    int* result = new int[size];
    for (int i = 0, j = size - 1; i < size; i++, j--)
    {
        result[j] = list[i];
    }
    return result;
}
void printArray(const int* list, int size)
{
    for (int i = 0; i < size; i++)
        cout << list[i] << " ";
}
int main()
{
    int list[] = {1, 2, 3, 4, 5, 6};
    printArray(list, 6);
    cout << endl;
    int* p = reverse(list, 6);
    printArray(p, 6);
    cout << endl;
    return 0;
}
\end{lstlisting}

\subsection*{The Output}
\begin{figure}[H]
    \centering
    \includegraphics{Assignmant_2_D}
    \caption{This code works as intended}
    \label{fig:my_label}
\end{figure}

\subsection*{The Description}
This code works just as intended. It takes in an array, and is able to print it forwards and backwards. 


\end{document}
