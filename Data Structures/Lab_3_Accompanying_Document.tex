\documentclass[11pt]{article}

\usepackage[utf8]{inputenc}
\usepackage{listings}
\usepackage{color}
\usepackage{float}

\usepackage{geometry}
\geometry{a4paper}
\geometry{margin = .5in}

\usepackage{graphicx}

\definecolor{dkgreen}{rgb}{0,0.6,0}
\definecolor{gray}{rgb}{0.5,0.5,0.5}
\definecolor{mauve}{rgb}{0.58,0,0.82}
\lstset{frame=tb,
  language=C++,
  aboveskip=3mm,
  belowskip=3mm,
  showstringspaces=false,
  columns=flexible,
  basicstyle={\small\ttfamily},
  numbers=none,
  numberstyle=\tiny\color{gray},
  keywordstyle=\color{blue},
  commentstyle=\color{dkgreen},
  stringstyle=\color{mauve},
  breaklines=true,
  breakatwhitespace=true,
  tabsize=3
}


\title{Lab 3 Accompanying Document}
\author{Harry Haisty}

\begin{document}

\maketitle

\section*{The Output}
\begin{figure}[H]
\centering

\includegraphics[width=15cm, height = 15cm,keepaspectratio]{lab3_1}
\caption{Printed output for lab 3}
\end{figure}

\newpage
\section*{The Code}

\subsection*{bag1.h}
\begin{lstlisting}
// FILE: bag1.h
// CLASS PROVIDED: bag (part of the namespace main_savitch_3)
//
// TYPEDEF and MEMBER CONSTANTS for the bag class:
//   typedef ____ value_type
//     bag::value_type is the data type of the items in the bag. It may be any of
//     the C++ built-in types (int, char, etc.), or a class with a default
//     constructor, an assignment operator, and operators to
//     test for equality (x == y) and non-equality (x != y).
//
//   typedef ____ size_type
//     bag::size_type is the data type of any variable that keeps track of how many items
//     are in a bag.
//
//   static const size_type CAPACITY = _____
//     bag::CAPACITY is the maximum number of items that a bag can hold.
//
// CONSTRUCTOR for the bag class:
//   bag( )
//     Postcondition: The bag has been initialized as an empty bag.
//
// MODIFICATION MEMBER FUNCTIONS for the bag class:
//   size_type erase(const value_type& target);
//     Postcondition: All copies of target have been removed from the bag.
//     The return value is the number of copies removed (which could be zero).
//
//   void erase_one(const value_type& target)
//     Postcondition: If target was in the bag, then one copy has been removed;
//     otherwise the bag is unchanged. A true return value indicates that one
//     copy was removed; false indicates that nothing was removed.
//
//   void insert(const value_type& entry)
//     Precondition:  size( ) < CAPACITY.
//     Postcondition: A new copy of entry has been added to the bag.
//
//   void operator +=(const bag& addend)
//     Precondition:  size( ) + addend.size( ) <= CAPACITY.
//     Postcondition: Each item in addend has been added to this bag.
//
// CONSTANT MEMBER FUNCTIONS for the bag class:
//   size_type size( ) const
//     Postcondition: The return value is the total number of items in the bag.
//
//   size_type count(const value_type& target) const
//     Postcondition: The return value is number of times target is in the bag.
//
// NONMEMBER FUNCTIONS for the bag class:
//   bag operator +(const bag& b1, const bag& b2)
//     Precondition:  b1.size( ) + b2.size( ) <= bag::CAPACITY.
//     Postcondition: The bag returned is the union of b1 and b2.
//
// VALUE SEMANTICS for the bag class:
//    Assignments and the copy constructor may be used with bag objects.

#ifndef MAIN_SAVITCH_BAG1_H
#define MAIN_SAVITCH_BAG1_H

#include <cstdlib>  // Provides size_t

namespace main_savitch_3 {
    class bag {
    public:
        // TYPEDEFS and MEMBER CONSTANTS
        typedef int value_type;
        typedef std::size_t size_type;
        static const size_type CAPACITY = 30;

        // CONSTRUCTOR
        bag() { used = 0; }

        // MODIFICATION MEMBER FUNCTIONS
        size_type erase(const value_type &target);

        // Takes in user value and removes value from list if it exists
        bool erase_one(const value_type &target);

        // Takes in value and adds it to the list
        void insert(const value_type &entry);

        void operator+=(const bag &addend);

        // CONSTANT MEMBER FUNCTIONS
        size_type size() const { return used; }

        size_type count(const value_type &target) const;

    private:
        value_type data[CAPACITY];  // The array to store items
        size_type used;             // How much of array is used
    };

    // NONMEMBER FUNCTIONS for the bag class
    bag operator+(const bag &b1, const bag &b2);
}

#endif
\end{lstlisting}


\subsection*{bag1.cpp}
\begin{lstlisting}
// FILE: bag1.cpp
// CLASS IMPLEMENTED: bag (see bag1.h for documentation)
// INVARIANT for the bag class:
//   1. The number of items in the bag is in the member variable used;
//   2. For an empty bag, we do not care what is stored in any of data; for a
//      non-empty bag the items in the bag are stored in data[0] through
//      data[used-1], and we don't care what's in the rest of data.

#include <algorithm> // Provides copy function
#include <cassert>   // Provides assert function
#include "bag1.h"

using namespace std;

namespace main_savitch_3 {
    const bag::size_type bag::CAPACITY;

    bag::size_type bag::erase(const value_type &target) {
        size_type index = 0;
        //set default value of removed items to 0
        size_type many_removed = 0;

        while (index < used) {
            if (data[index] == target) {
                --used;
                data[index] = data[used];
                ++many_removed;
            } else
                ++index;
        }
        return many_removed;
    }

    bool bag::erase_one(const value_type &target) {
        size_type index; // The location of target in the data array

        // First, set index to the location of target in the data array,
        // which could be as small as 0 or as large as used-1. If target is not
        // in the array, then index will be set equal to used.
        index = 0;
        //navigate through indices to find the target to remove
        while ((index < used) && (data[index] != target))
            ++index;

        if (index == used)
            return false; // target isn’t in the bag, so no work to do.

        // When execution reaches here, target is in the bag at data[index].
        // So, reduce used by 1 and copy the last item onto data[index].
        --used;
        data[index] = data[used];
        return true;
    }

    void bag::insert(const value_type &entry)
    // Library facilities used: cassert
    {
        //compares current size to the final value of absolute capacity
        assert(size() < CAPACITY);

        //sets entry in specific place
        data[used] = entry;
        //increments amount of "used" space
        ++used;
    }

    void bag::operator+=(const bag &addend)
    // Library facilities used: algorithm, cassert
    {
        //checks to see if current size and future size is less than final value capacity
        assert(size() + addend.size() <= CAPACITY);

        copy(addend.data, addend.data + addend.used, data + used);
        used += addend.used;
    }

    bag::size_type bag::count(const value_type &target) const {
        size_type answer;
        size_type i;

        //sets default value to 0 so returned value cannot be null
        answer = 0;
        for (i = 0; i < used; ++i)
            if (target == data[i])
                //increments answer value with each value found
                ++answer;
            //returns value found
        return answer;
    }

    bag operator+(const bag &b1, const bag &b2)
    // Library facilities used: cassert
    {
        bag answer;

        //compares the size of the two bags to the size of the final constant capacity
        assert(b1.size() + b2.size() <= bag::CAPACITY);

        answer += b1;
        answer += b2;
        return answer;
    }
}

\end{lstlisting}

\subsection*{bag1\_demo.cpp}
\begin{lstlisting}
// FILE: bag_demo.cpp
// This is a small demonstration program showing how the bag class is used.
#include <iostream>    // Provides cout and cin
#include <cstdlib>     // Provides EXIT_SUCCESS
#include "bag1.h"      // With value_type defined as an int

using namespace std;
using namespace main_savitch_3;

// PROTOTYPES for functions used by this demonstration program:
void get_ages(bag &ages);
// Postcondition: The user has been prompted to type in the ages of family
// members. These ages have been read and placed in the ages bag, stopping
// when the bag is full or when the user types a negative number.

void check_ages(bag &ages);
// Postcondition: The user has been prompted to type in the ages of family
// members once again. Each age is removed from the ages bag when it is typed,
// stopping when the bag is empty.

void is_full(bag &ages);

int main() {
    bag ages;

    get_ages(ages);
    check_ages(ages);
    cout << "May your family live long and prosper." << endl;
    return EXIT_SUCCESS;
}

void get_ages(bag &ages) {
    int user_input;

    cout << "Type the ages in your family." << endl;
    cout << "Type a negative number when you are done:" << endl;
    cin >> user_input;
    while (user_input >= 0) {
        if (ages.size() < ages.CAPACITY) {
            ages.insert(user_input);
            is_full(ages);
        } else
            cout << "I have run out of room and can’t add that age." << endl;
        cin >> user_input;
    }
}

//My code, checks to see if there is space left in the bag and return that value to user
void is_full(bag &ages) {
    //compares the size of the capacity, which is 30, to the size of the current bag
    if (ages.size() < ages.CAPACITY) {
        //subtract value of bag from final value capacity
        int room_left = ages.CAPACITY - ages.size();
        cout << "There are " << room_left << " spaces left." << endl;
    } else
        //shows user that bag is at capacity
        cout << "Bag is full!" << endl;
}

void check_ages(bag &ages) {
    int user_input;

    //prompts user to re-enter values so that they may be removed from the bag
    cout << "Type those ages again. Press return after each age:" << endl;
    //stops removal from bag at value 0
    while (ages.size() > 0) {
        cin >> user_input;
        //implements erase one function to strike one particular value from bag
        if (ages.erase_one(user_input)) {
            cout << "Yes, I've got that age and will remove it." << endl;
            is_full(ages);
        }
        //makes sure that value actually exists in bag
        else {
            cout << "No, that age does not occur!" << endl;
            is_full(ages);
        }
    }
}
\end{lstlisting}

\section{Added Functionality}
I added the functionality to return the remaining spaces in the program to the user. This allows the user to realize how many values are left in the bag, and lets them determine how many values they would like to add to the program. Obviously, it's not going to happen very often that a user will put 30 values of family ages into the program, but I guess if they have a very big family it is entirely possible. 

\newpage
\section{The Final Output}
\begin{figure}[H]
\centering
\includegraphics[width=15cm, height = 15cm,keepaspectratio]{lab3_2}
\caption{Printed output for lab 3}
\end{figure}


\section{Questions}
\begin{enumerate}
    \item The functions are defined but not used in the original implementation. The main program calls functions that don't use all of the defined functions in the implementation program. 
    
    \item They are not all tested by the original program. The ones that are not tested are 
    \begin{lstlisting}
        bag::size_type bag::count(const value_type &target) const {
        size_type answer;
        size_type i;

        //sets default value to 0 so returned value cannot be null
        answer = 0;
        for (i = 0; i < used; ++i)
            if (target == data[i])
                //increments answer value with each value found
                ++answer;
            //returns value found
        return answer;
    }
    \end{lstlisting}
and...
    \begin{lstlisting}
        bag::size_type bag::erase(const value_type &target) {
        size_type index = 0;
        //set default value of removed items to 0
        size_type many_removed = 0;

        while (index < used) {
            if (data[index] == target) {
                --used;
                data[index] = data[used];
                ++many_removed;
            } else
                ++index;
        }
        return many_removed;
    }
    \end{lstlisting}
    
    \item The code cannot work without main\_savitch3 because the namespace includes all of the functions that are accessed in the methods defined in the test program. The program could be made to work if everything was defined in the test program, but then it would be very cumbersome and not very well-organized.
\end{enumerate}
\end{document}