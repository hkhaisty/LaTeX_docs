\documentclass[11pt]{article}

\usepackage[utf8]{inputenc}
\usepackage{listings}
\usepackage{color}
\usepackage{float}
\usepackage{xcolor}
\colorlet{Mycolor1}{green!10!orange!90!}

\usepackage{geometry}
\geometry{a4paper}
\geometry{margin = .5in}

\usepackage{graphicx}

\definecolor{dkgreen}{rgb}{0,0.6,0}
\definecolor{gray}{rgb}{0.5,0.5,0.5}
\definecolor{mauve}{rgb}{0.58,0,0.82}
\lstset{frame=tb,
  language=C++,
  aboveskip=3mm,
  belowskip=3mm,
  showstringspaces=false,
  columns=flexible,
  basicstyle={\small\ttfamily},
  numbers=none,
  numberstyle=\tiny\color{gray},
  keywordstyle=\color{Mycolor1},
  commentstyle=\color{dkgreen},
  stringstyle=\color{pink},
  breaklines=true,
  breakatwhitespace=true,
  tabsize=3
}

\title{Lab 2 Accompanying Document}
\author{Harry Haisty}

\begin{document}
\maketitle

\section{The Output}

\begin{figure}[H]
    \centering
    \includegraphics{Lab2_2}
\end{figure}


\section{The Code}
\begin{lstlisting}
#include <iostream>
using namespace std;

const int SIZE = 4;


void printArray(int list[], int arraySize)
{
    for (int i = 0; i < arraySize; i++)
    {
        cout << list[i] <<  " ";
    }
	cout << "\n";
}

void reverse(const int list[], int newList[], int size)
{
    for (int i = 0, j = size - 1; i < size; i++, j--)
    {
        newList[j] = list[i];
    }
}

void p(int list[], int arraySize)
{
    list[0] = {arraySize};
}

int main()
{
    SIZE;
    int newList[SIZE];
    int numbers[] = {1, 4, 3, 6, 8};
    p(numbers, 5);
    printArray(numbers, 5);
    reverse(numbers, newList, SIZE);
    printArray(newList, SIZE);

    return 0;
}
//Duplicate, we can copy all this info into the printArray in the top of the program
// void printArray(int list[], int arraySize)
// {
//     for (int i = 0; i < arraySize; i++)
//     {
//         cout << list[i] <<  " ";
//     }
// }


\end{lstlisting}


\section*{The Output}

\begin{figure}[H]
    \centering
    \includegraphics{lab2_1}
\end{figure}
\section*{The Code}

\subsection*{main.cpp}

\begin{lstlisting}
#include <iostream>
using namespace std;

void m(int, int []);
//remove const to make list accessible
void p(int list[], int arraySize)
{
    list[0] = 100;
}

int main()
{
    int x = 1;
    int y[10];
    y[0] = 1;

    m(x, y);

    cout << "x is " << x << endl;
    cout << "y[0] is " << y[0] << endl;
}

void m(int number, int numbers[])
{
    number = 1001;
    numbers[0] = 5555;
}
\end{lstlisting}

\subsection*{}
\begin{lstlisting}


\end{lstlisting}

\section*{The Output}

\begin{figure}[H]
    \centering
    \includegraphics{lab2_3}
\end{figure}

\section*{The Code}
\begin{lstlisting}
#include <iostream>
using namespace std;

//change integer types to match rest of program, otherwise it is an illegitimate cast
void swap1(int* n1, int* n2)
{
    int temp = *n1;
    *n1 = *n2;
    *n2 = temp;
}

void swap2(int& n1, int& n2)
{
    int temp = n1;
    n1 = n2;
    n2 = temp;
}

void swap3(int* p1, int* p2)
{
    int temp = *p1;
    *p1 = *p2;
    *p2 = temp;
}

void swap4(int* p1, int* p2)
{
    int temp = *p1;
    *p1 = *p2;
    *p2 = temp;
}

int main()
{
    int num1 = 1;
    int num2 = 2;

    cout << "Before invoking the swap1 function, num1 is "
         << num1 << " and num2 is " << num2 << endl;

    swap1(&num1, &num2);

    cout << "After invoking the swap1 function, num1 is " << num1 <<
         " and num2 is " << num2 << endl;

    cout << "Before invoking the swap2 function, num1 is "
         << num1 << " and num2 is " << num2 << endl;

    swap2(num1, num2);

    cout << "After invoking the swap2 function, num1 is " << num1 <<
         " and num2 is " << num2 << endl;

    cout << "Before invoking the swap3 function, num1 is "
         << num1 << " and num2 is " << num2 << endl;

    swap3(&num1, &num2);

    cout << "After invoking the swap3 function, num1 is " << num1 <<
         " and num2 is " << num2 << endl;

    int p1 = num1;
    int p2 = num2;
    cout << "Before invoking the swap4 function, p1 is "
         << p1 << " and p2 is " << p2 << endl;

//add & sign so that compiler reads numerical value instead of address location
    swap4(&p1, &p2);

    cout << "After invoking the swap4 function, p1 is " << p1 <<
         " and p2 is " << p2 << endl;

    return 0;
}

\end{lstlisting}

\section{The Output}
I could not complete this challenge

\section{The Code}
\subsection{thinker.cpp}
\begin{lstlisting}
#include <iostream>
#include <stdlib.h>
#include <assert.h>
#include <cstring>
#include "thinker.h"

using namespace std;

void thinking_cap::slots(char new_green[], char new_red[]) {
    assert(strlen(new_green) < 50);
    assert(strlen(new_red) < 50);
    strcpy(green_string, new_green);
    strcpy(red_string, new_red);
}

void thinking_cap::push_green()
        {
                cout << green_string << endl;
        };
void thinking_cap::push_red()
        {
                cout << red_string << endl;
        };

\end{lstlisting}

\subsection{TestThinker.cpp}
\begin{lstlisting}
#include "thinker.h"

int main( )
{
    thinking_cap student;
    thinking_cap fan;
    student.slots( c_str("Hello"),  c_str("Goodbye"));
    fan.slots( "Go Cougars!", "Boo!");
    student.push_green( );
    fan.push_green( );
    student.push_red( );
    return 0;
}

\end{lstlisting}

\subsection{thinker.h}
\begin{lstlisting}
class thinking_cap
{
public:
    void slots(char new_green[ ], char new_red[ ]);
    void push_green();
    void push_red();
private:
    static char green_string[50];
    static char red_string[50];
};
\end{lstlisting}

\end{document}