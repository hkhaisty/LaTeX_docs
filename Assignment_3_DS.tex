\documentclass[11pt]{article}
\usepackage[utf8]{inputenc}
\usepackage{listings}
\usepackage{color}
\usepackage{float}
\usepackage{geometry}
\geometry{a4paper}
\geometry{margin = .5in}

\usepackage{graphicx}

\definecolor{dkgreen}{rgb}{0,0.6,0}
\definecolor{gray}{rgb}{0.5,0.5,0.5}
\definecolor{mauve}{rgb}{0.58,0,0.82}
\lstset{
  language=C++,
  aboveskip=3mm,
  belowskip=3mm,
  showstringspaces=false,
  columns=flexible,
  basicstyle={\small\ttfamily},
  numbers=none,
  numberstyle=\tiny\color{gray},
  keywordstyle=\color{mauve},
  commentstyle=\color{dkgreen},
  stringstyle=\color{blue},
  breaklines=true,
  breakatwhitespace=true,
  tabsize=3
}


\title{Assignment 3 Accompanying Document}
\author{Harry Haisty}
\date{November 2018}

\begin{document}

\maketitle

\section*{Section 3.2}
\begin{enumerate}
    %%%%%%%%%%%
    
    \item[a.] 
    \begin{lstlisting}
sum = 0;
for (int i = 0; i < n; i++)
  for (int j = 0; j < n * n; j++)
    sum++;
    \end{lstlisting}
    
    \item[answer:] This code runs $n^2$ times.
    %%%%%%%%%%%%
    
    \item[b.]
    \begin{lstlisting}
long fib (long index)
{
  if (index == 0)		// Base case
    return 0;
  else if (index == 1)		// Base case
    return 1;
  else				// Reduction and recursive calls
    return fib (index b - 1) + fib (index b - 2);
}
    \end{lstlisting}
    
    %%%%%%%%%%%%%%
    
    \item[c.]
    \begin{lstlisting}
int gcd (int m, int n)
{
    if (m % n == 0)
        return n;
    else
       return gcd (n, m % n);
}
    \end{lstlisting}
    
    %%%%%%%%%%%%%%%
    
    \item[d.]
    \begin{lstlisting}
sum = 0;
for (int i = 0; i < n; i++)
  for (int j = 0; j < i * i; j++)
    for (int k = 0; k < j; k++)
  sum++;
    \end{lstlisting}
    
    \item[answer:] This code runs $n^5$ times.
    
    %%%%%%%%%%%%%%%%%%
    
    \item[e.]
    \begin{lstlisting}
sum = 0;
for (int i = 1; i < n; i++)
  for (int j = 1; j < i * i; j++)
    if (j % i == 0)
  for (k = 0; k < j; k++)
    sum++;
    \end{lstlisting}
    
    \item[answer:] This code runs $n^4$ times.
    
\end{enumerate}

\section*{Section 3.4}
\begin{figure}[h]
    \centering
    \includegraphics[width=7cm]{treepart1}
    \caption{Original tree}
    \label{fig:my_label}
\end{figure}

\begin{figure}[h]
    \centering
    \includegraphics[width=7cm]{treepart2}
    \caption{After 10 is added to tree}
    \label{fig:my_label}
\end{figure}

\begin{figure}[h]
    \centering
    \includegraphics[width=7cm]{treepart3}
    \caption{After 15 is removed from tree}
    \label{fig:my_label}
\end{figure}

\section*{Section 3.5}
\begin{figure}
    \centering
    \includegraphics[width = 11cm]{binarytreewithd}
    \caption{Caption}
    \label{fig:my_label}
\end{figure}



\section*{Section 3.6}
I used the \textbf{Merge Sort} algorithm to sort this data set.
\begin{figure}
    \centering
    \includegraphics[width= 7cm]{mergesort}
    \caption{My merge sort figure for the data set}
    \label{fig:my_label}
\end{figure}

\section*{Section 3.7}

\begin{enumerate}

\item List all the paths from A to H 
\begin{enumerate}
    \item[] A, B, D, H
    \item[] A, B, D, F, H
    \item[] A, C, F, H
    \item[] A, C, B, D, H
    \item[] A, E, B, D, H
    \item[] A, E, G, H
\end{enumerate}

\item which paths have the lowest weight?
\begin{enumerate}
    \item[] A, C, F, H
    \item[] A, B, D, F, H
    \item[] A, E, G, H
\end{enumerate}
These paths all have a weight of 31. 

\item Which path has the shortest length?
\begin{enumerate}
    \item[] A, C, F, H
    \item[] A, B, D, H
    \item[] A, E, G, H
\end{enumerate}
These paths all are of length 4.

\item Is the graph connected strongly or weakly? Explain. \newline 
It is connected weakly because the paths only go one direction, which means that there is no connecting path between any two nodes.  

\end{enumerate}

\section*{Section 3.8}
\begin{enumerate}
    \item a breadth-first search \newline
    A, B, C, D, E, F
    
    \item a depth-first search \newline
    A, B, E, C, F, D
    
    \item make an adjacency list
    
    \item make an adjacency matrix
    \begin{figure}
        \centering
        \includegraphics[width = 8cm]{adjacencymatrix}
        \caption{Adjacency matrix for 3.8}
        \label{fig:my_label}
    \end{figure}
    \item make an incidence matrix for this graph
    \begin{figure}
        \centering
        \includegraphics[width = 8cm]{incidencematrix}
        \caption{Incidence matrix for this graph}
        \label{fig:my_label}
    \end{figure}
    
    \item identify a cycle in the graph \newline
    There is a cycle between vertices A, B, and C.
    
    \item Is the graph complete? Explain. \newline
    This graph is not complete, there are not paths between each node.
    
    \begin{figure}
        \centering
        \includegraphics[width=6cm]{completegraph}
        \caption{The red lines represent the missing paths in order to make this graph "complete"}
        \label{fig:my_label}
    \end{figure}
\end{enumerate}

\begin{figure}
    \centering
    \includegraphics[width = 11cm]{abstractsubmission}
    \caption{My 2019 NCUR abstract submission}
    \label{fig:my_label}
\end{figure}


\end{document}
