\documentclass[11pt]{article}
\usepackage[utf8]{inputenc}
\usepackage{listings}
\usepackage{color}
\usepackage{float}
\usepackage{geometry}
\geometry{a4paper}
\geometry{margin = .5in}

\usepackage{graphicx}

\definecolor{dkgreen}{rgb}{0,0.6,0}
\definecolor{gray}{rgb}{0.5,0.5,0.5}
\definecolor{mauve}{rgb}{0.58,0,0.82}
\lstset{
  language=C++,
  aboveskip=3mm,
  belowskip=3mm,
  showstringspaces=false,
  columns=flexible,
  basicstyle={\small\ttfamily},
  numbers=none,
  numberstyle=\tiny\color{gray},
  keywordstyle=\color{mauve},
  commentstyle=\color{dkgreen},
  stringstyle=\color{blue},
  breaklines=true,
  breakatwhitespace=true,
  tabsize=3
}


\title{Assignment 3 Accompanying Document}
\author{Harry Haisty}
\date{November 2018}

\begin{document}

\maketitle

\section*{Section 3.1}

\begin{enumerate}
    \item \begin{enumerate}
    \item[a.] $\{1, 2, 3, 5\}$
    \item[b.] $\{3\}$
    \item[c.] $\{2,4\}$
    \item[d.] $\{1\}$
    \item[e.] $\{5, 2\}$
    \item[f.] No, Z does not contain the value 1.
\end{enumerate}

    \item \begin{enumerate}
        \item
    \end{enumerate}
\end{enumerate}



\section*{Section 3.2}
\begin{enumerate}
    %%%%%%%%%%%
    
    \item[a.] 
    \begin{lstlisting}
sum = 0;
for (int i = 0; i < n; i++)
  for (int j = 0; j < n * n; j++)
    sum++;
    \end{lstlisting}
    
    \item[answer:] This code runs $n^2$ times.
    %%%%%%%%%%%%
    
    \item[b.]
    \begin{lstlisting}
long fib (long index)
{
  if (index == 0)		// Base case
    return 0;
  else if (index == 1)		// Base case
    return 1;
  else				// Reduction and recursive calls
    return fib (index b - 1) + fib (index b - 2);
}
    \end{lstlisting}
    
    %%%%%%%%%%%%%%
    
    \item[c.]
    \begin{lstlisting}
int gcd (int m, int n)
{
    if (m % n == 0)
        return n;
    else
       return gcd (n, m % n);
}
    \end{lstlisting}
    
    %%%%%%%%%%%%%%%
    
    \item[d.]
    \begin{lstlisting}
sum = 0;
for (int i = 0; i < n; i++)
  for (int j = 0; j < i * i; j++)
    for (int k = 0; k < j; k++)
  sum++;
    \end{lstlisting}
    
    \item[answer:] This code runs $n^5$ times.
    
    %%%%%%%%%%%%%%%%%%
    
    \item[e.]
    \begin{lstlisting}
sum = 0;
for (int i = 1; i < n; i++)
  for (int j = 1; j < i * i; j++)
    if (j % i == 0)
  for (k = 0; k < j; k++)
    sum++;
    \end{lstlisting}
    
    \item[answer:] This code runs $n^4$ times.
    
\end{enumerate}

\section*{Section 3.4}
\begin{figure}[h]
    \centering
    \includegraphics[width=7cm]{treepart1}
    \caption{Original tree}
    \label{fig:my_label}
\end{figure}

\begin{figure}[h]
    \centering
    \includegraphics[width=7cm]{treepart2}
    \caption{After 10 is added to tree}
    \label{fig:my_label}
\end{figure}

\begin{figure}[h]
    \centering
    \includegraphics[width=7cm]{treepart3}
    \caption{After 15 is removed from tree}
    \label{fig:my_label}
\end{figure}

\section*{Section 3.5}
\begin{figure}
    \centering
    \includegraphics[width = 11cm]{binarytreewithd}
    \caption{Binary tree with node D added}
    \label{fig:my_label}
\end{figure}

\begin{figure}
    \centering
    \includegraphics[width = 11cm]{rearrangetree}
    \caption{Reordered binary tree}
    \label{fig:my_label}
\end{figure}


\section*{Section 3.6}
I used the \textbf{Merge Sort} algorithm to sort this data set.
\begin{figure}
    \centering
    \includegraphics[width= 7cm]{mergesort}
    \caption{My merge sort figure for the data set}
    \label{fig:my_label}
\end{figure}

\section*{Section 3.7}

\begin{enumerate}

\item List all the paths from A to H 
\begin{enumerate}
    \item[] A, B, D, H
    \item[] A, B, D, F, H
    \item[] A, C, F, H
    \item[] A, C, B, D, H
    \item[] A, E, B, D, H
    \item[] A, E, G, H
\end{enumerate}

\item which paths have the lowest weight?
\begin{enumerate}
    \item[] A, C, F, H
    \item[] A, B, D, F, H
    \item[] A, E, G, H
\end{enumerate}
These paths all have a weight of 31. 

\item Which path has the shortest length?
\begin{enumerate}
    \item[] A, C, F, H
    \item[] A, B, D, H
    \item[] A, E, G, H
\end{enumerate}
These paths all are of length 4.

\item Is the graph connected strongly or weakly? Explain. \newline 
It is connected weakly because the paths only go one direction, which means that there is no connecting path between any two nodes.  

\end{enumerate}

\section*{Section 3.8}
\begin{enumerate}
    \item a breadth-first search \newline
    A, B, C, D, E, F
    
    \item a depth-first search \newline
    A, B, E, C, F, D
    
    \item make an adjacency list
    
    \item make an adjacency matrix
    \begin{figure}
        \centering
        \includegraphics[width = 8cm]{adjacencymatrix}
        \caption{Adjacency matrix for 3.8}
        \label{fig:my_label}
    \end{figure}
    \item make an incidence matrix for this graph
    \begin{figure}
        \centering
        \includegraphics[width = 8cm]{incidencematrix}
        \caption{Incidence matrix for this graph}
        \label{fig:my_label}
    \end{figure}
    
    \item identify a cycle in the graph \newline
    There is a cycle between vertices A, B, and C.
    
    \item Is the graph complete? Explain. \newline
    This graph is not complete, there are not paths between each node.
    
    \begin{figure}
        \centering
        \includegraphics[width=6cm]{completegraph}
        \caption{The red lines represent the missing paths in order to make this graph "complete"}
        \label{fig:my_label}
    \end{figure}
\end{enumerate}

\section*{Section 3.9}
\begin{figure}
    \centering
    \includegraphics[width = 11cm]{abstractsubmission}
    \caption{My 2019 NCUR abstract submission}
    \label{fig:my_label}
\end{figure}

I couldn't make it to either of the abstract writing workshops, but I was able to get in touch with the people in the Technical Communication department here at KSU. They gave me tons of really good writing tips, and helped me edit my abstract so that I could submit it. I have submitted it to NCUR, ACM, and IEEE. 

\section*{Section 3.10}

\subsection*{The Code}
\begin{lstlisting}

//methods.h
using namespace std;

class mixed_number_operations {

public :
    float regular_number, complex_number;

    void printNumber() {
        if (complex_number > 0) {
            cout << regular_number << "+" << complex_number << "i";
        } else {
            cout << regular_number << "" << complex_number << "i";
        }
    }

    void add_numbers(mixed_number_operations first, mixed_number_operations second) {
        regular_number = first.regular_number + second.regular_number;
        complex_number = first.complex_number + second.complex_number;
        if (complex_number > 0) {
            cout << regular_number << "+" << complex_number << "i";
        } else {
            cout << regular_number << "" << complex_number << "i";
        }
    }

    void subtract_numbers(mixed_number_operations one, mixed_number_operations two) {
        regular_number = one.regular_number - two.regular_number;
        complex_number = one.complex_number - two.complex_number;
        if (complex_number > 0) {
            cout << setprecision(2);
            cout << regular_number << "+" << complex_number << "i";
        } else {
            cout << setprecision(2);
            cout << regular_number << complex_number << "i";
        }
    }

    void multiply_numbers(mixed_number_operations one, mixed_number_operations two) {
        regular_number = one.regular_number * two.regular_number;
        complex_number = one.complex_number * two.complex_number;
        if (complex_number > 0) {
            cout << regular_number << "+" << complex_number << "i";
        } else {
            cout << regular_number << "" << complex_number << "i";
        }
    }

    void divide_numbers(mixed_number_operations one, mixed_number_operations two) {
        if (two.regular_number == 0 || two.complex_number == 0) {
            cout << "Sorry division by 0 is not possible";
        } else {
            regular_number = one.regular_number / two.regular_number;
            complex_number = one.complex_number / two.complex_number;
            if (complex_number > 0) {
                cout << regular_number << "+" << complex_number << "i";
            } else {
                cout << regular_number << "" << complex_number << "i";
            }
        }
    }

    void printing_results(mixed_number_operations first_value, mixed_number_operations second_value) {

        cout << endl << "1. The FIRST complex number is: "; first_value.
                printNumber();

        cout << endl << "2. The SECOND complex number is: "; second_value.
                printNumber();

        cout << endl << "3. The SUM of the complex numbers is: ";
                add_numbers(first_value, second_value);

        cout << endl << "4. The MULTIPLICATION of the complex numbers is: ";
                multiply_numbers(first_value, second_value);

        cout << endl << "5. The SUBTRACTION of the complex number is: ";
                subtract_numbers(first_value, second_value);

        cout << endl << "6. The DIVISION of the complex numbers is: ";
                divide_numbers(first_value, second_value);

        cout << endl;
    }
};

//main.cpp

#include <iostream>
#include <math.h>
#include<iomanip>
#include "methods.h"

using namespace std;

int main() {
    mixed_number_operations first_input_value, second_input_value;

    //prompts user to enter integer part of rational number
    cout << "Enter integer part of first normal number: ";
    cin >> first_input_value.regular_number;

    //prompts user to enter integer part of complex number (minus i)
    cout << "Enter integer value of complex number(*not* including the letter i): ";
    cin >> first_input_value.complex_number;

    //prompts user to enter integer part of second rational number
    cout << "Enter integer part of second normal number: ";
    cin >> second_input_value.regular_number;

    //prompts uer to enter integer part of second complex number (minus i)
    cout << "Enter integer part of second complex number (*not* including the letter i): ";
    cin >> second_input_value.complex_number;

    //calls printing method to display results of method calls.
    first_input_value.printing_results(first_input_value, second_input_value);
}

\end{lstlisting}

\begin{figure}[h]
    \centering
    \includegraphics[width = 12cm]{310result}
    \caption{code output for section 3.10}
    \label{fig:my_label}
\end{figure}


\section*{3.12}

\subsection*{The Code}

\begin{lstlisting}
//deque Demo
#include <iostream>
#include <list>
#include <algorithm>
#include <deque>
#include <string>

using namespace std;

//creates standard template
template<class T>

void printDeque(const deque<T> &lst, string s) {
    //prints formatting for output
    cout << s << ": ";
    //calls generic template type at first index
    typename deque<T>::const_iterator i = lst.begin();
    //iterates through templated input
    for (; i != lst.end(); i++)
        //outputs addressed value
        cout << *i << ' ';
    cout << endl;
}

int main() {
    //call template
    deque<int> dq1;

    //push value 1 to front of deque {1}
    dq1.push_front(1);

    //push value 2 to front of deque {2, 1}
    dq1.push_front(2);

    //push value 3 to back of deque {2, 1, 3}
    dq1.push_back(3);

    //push value 4 to back of deque {2, 1, 3, 4}
    dq1.push_back(4);

    //prints out deque so far and "dql" identifier
    printDeque(dq1, "dq1");

    //calls generic template type with int
    deque<int> dq2(dq1.begin() + 1, dq1.end() - 1);

    //sets value at index 1 to 5
    dq1[1] = 5;

    //erases value at beginning of deque
    dq1.erase(dq1.begin());

    //adds new elements at specified positions
    dq1.insert(dq1.end() - 1, 2, 6);

    //calls sort function for deque
    sort(dq1.begin(), dq1.end());

    //calls generic template with type int
    deque<int> dq3;

    //resizes dq3 so that it contains the number of elements in dq1 and dq2 combined
    dq3.resize(dq1.size() + dq2.size());

    //merges all three dq's together
    merge(dq1.begin(), dq1.end(), dq2.begin(), dq2.end(), dq3.begin());

    //prints values of deques
    printDeque(dq1, "dq1");
    printDeque(dq2, "dq2");
    printDeque(dq3, "dq3");
    return 0;
}
\end{lstlisting}

\begin{figure}[h]
    \centering
    \includegraphics[width = 11cm]{312Output}
    \caption{Output for section 3.12}
    \label{fig:my_label}
\end{figure}

\end{document}
